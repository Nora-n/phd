\documentclass[../main/main.tex]{subfiles}
\begin{document}

\chapter*{Abstract}\label{ch}

Type Ia Supernovae (SNe~Ia) are celestial bodies of transient luminosity
resulting from the explosion of stars. They are nowadays at the heart of
observational cosmology analysis because of their regularity in their released
luminosity, which allows different surveys and telescopes to measure their
distance and thus the expansion rate of the Universe.

However, the detailed nature of SNe~Ia remains uncertain, and these studies rely
on empirical laws and in particular on the distinction into two populations of
SNe~Ia which would have different properties. As the statistics of the surveys
increase, the question of astrophysical systematic uncertainties arises,
including the evolution of SNe~Ia populations.

In this perspective, we implement attempts to improve our knowledge of the
physics of SNe~Ia through the study of correlations between their properties and
their environment. We have shown the existence of a bias related to the global
mass of the host galaxy of a SN, and highlighted the exitence of age-based
subpopulations that could be more relevant as a tracer of the difference in
properties observed in SNe.

Our thesis builds on this hypothesis and the link established by previous
studies between SNe stretch and age. In this thesis, we study the redshift
dependence of the light curve stretch from a \texttt{SALT2} fit of SNe Ia, which
is a purely intrinsic property of SNe, to probe its potential drift with
redshift. We model different dependencies and give the results of our analysis:
we will see that the astrophysical drift of SNe~Ia properties is strongly
favored and that the underlying distribution models of constant stretches with
redshift are excluded as good representations of the data with respect to our
reference model.

The impact of this modeling on the determination of cosmological parameters has
been studied through numerical simulations, and indicate a bias of up to 4\% in
the value of the dark energy state parameter, $w$, if these correlations are not
taken into account.

\end{document}
