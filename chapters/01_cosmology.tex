\documentclass[../main/main.tex]{subfiles}
\begin{document}

\chapter{Contexte cosmologique}\label{ch:cosmo}

\epigraph{Time is an illusion, a construct made out of human memory}{Blake
\textsc{Crouch}, \textit{Revelations}}

\noindent Bien que la cosmologie ne s'en tienne pas aux concepts récents tels
qu'on les connaît et les vulgarise, c'est avec les travaux d'\textsc{Einstein}
au début du XX\ieme~siècle que notre compréhension du monde cosmique prend son
essor. 

%\dominitoc
%\tableofcontents
\minitoc
\newpage

\section{Bases de relativité générale}\label{sec:11}

\subsection{Concepts initiaux}\label{ssec:RG}

\subsection{Métrique et équations de conservation}\label{ssec:112}

\subsection{Définition de la constante cosmologique}\label{ssec:lambda}

\section{Introduction du modèle standard de la cosmologie}\label{sec:MS}

\subsection{Univers plat, homogène et isotrope}\label{ssec:plat}

\paragraph*{Univers plat}
\lipsum[1]

\subsection{Métrique de Friedmann-Lemaître-Robertson-Walker}\label{ssec:FLRW}

\subsection{Le modèle \lcdm}\label{ssec:LCDM}

\section{Mesure cosmologiques}\label{sec:dist}

\subsection{Âge de l'Univers}\label{ssec:age}

\subsection{Distance de luminosité}\label{ssec:dl}

\subsection{Intérêt des supernovae de type Ia}\label{ssec:intsne}

\lipsum[2-4]

\end{document}
