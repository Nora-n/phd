\documentclass[../main/main.tex]{subfiles}
\begin{document}
\mainmatter

% \dominitoc
% \faketableofcontents
% \dominilof
% \fakelistoffigures
% \dominilot
% \fakelistoftables

\chapter{Contexte cosmologique}\label{ch:cosmo}

\epigraph{\openquote\textit{Time is an illusion, a construct made out of human
memory.}\closequote}{Blake \textsc{Crouch}, \textit{Revelations}}

Bien que la cosmologie ne s'en tienne pas aux concepts récents tels que nous les
connaissons et les vulgarisons, c'est avec les travaux d'\textsc{Einstein} au
début du XX\ieme~siècle que notre compréhension du monde cosmique se développe
de manière vertigineuse. Ces découvertes ont radicalement changé notre
appréhension de la structure et de l'histoire de notre Univers, dans la
continuité de la révolution copernicienne nous retirant du centre du monde~:
avec les fondements de la Relativité Générale, il n'y a non seulement aucun
point privilégié dans l'Univers, mais le temps même devient relatif à
l'observation.

Dans ce chapitre, nous donnons quelques notions et éléments de contexte
permettant de comprendre comment les SNe~Ia s'intègrent à la cosmologie
observationnelle moderne. Ainsi, dans la première section (\ref{sec:bases}) nous
discutons des bases nécessaires de relativité générale, ce qui nous mènera à
présenter le modèle cosmologique le plus robuste et qui constitue la base de
notre description du monde aujourd'hui. Ensuite, nous verrons dans la seconde
section de ce chapitre (\ref{sec:dist}) quelles sont les grandeurs d'intérêt à
la cosmologie observationnelle et comment les SNe~Ia permettent d'agrandir notre
connaissance sur l'Univers.

\vfill
\minitoc
\vfill

\newpage

\section{Bases de relativité générale}\label{sec:bases}

Nous introduisons dans cette section les idées et grandeurs fondamentales à la
description de l'Univers par la relativité générale. Elle n'a pas pour vocation
de détailler intégralement les idées de mécaniques classiques sur lesquelles
elles se basent, mais de donner une idée des changements qu'elle implique.

\subsection{Concepts initiaux}\label{ssec:RG}

La relativité générale augmente les idées déjà introduites par la relativité
restreinte \citep[voir][et les autres articles de l'\textit{annus
mirabilis}\footnote{C'est ainsi que nous nommons l'année 1905 pendant laquelle
    \textsc{Einstein} publie quatre articles considérés comme étant à l'origine
de la physique moderne.}]{einstein1905}, à savoir~:
\begin{enumerate}
    \item Les lois de la physique (\textit{hormis la gravitation}) sont
        invariantes par changement de référentiel galiléen\footnote{En quelques
            mots, nous pouvons traduire un référentiel comme an observataire,
            c'est-à-dire un lieu de mesure de distances et de durées de
            référence~; il est dit «~galiléen~», ou «~inertiel~», quand il n'est
            pas en accélération~: il peut l'être à une vitesse constante par
            rapport à un autre référentiel, c'est le cas d'un train à vitesse
        fixe et du paysage qu'il traverse.}~;
    \item La vitesse de la lumière dans le vide est identique dans tous les
        référentiels galiléens.
\end{enumerate}

Ces hypothèses, qui n'ont peut-être \textit{a priori} pas l'air
révolutionnaires, sont le fondement de notre conception du monde physique
actuel. D'une part, aucun point et aucune direction de l'espace n'est
particulière dans le cadre du fonctionnement des lois de la
physique\footnote{Nous disons que les lois de la physique sont
\textit{homogènes} et \textit{isotropes}, respectivement.} (l'Univers entier même
n'a donc pas de centre~!), et d'autre part pour chaque observataire la mesure de
la vitesse de la lumière sera la même, peu importe sa propre vitesse. C'est
subtilement affirmer que cette vitesse est infranchissable, définissant des
espaces qui sont non causalement reliés entre eux (ils ne peuvent interagir l'un
avec l'autre) et que les lois physiques sont décrites en quatre dimensions
(trois d'espace et une de temps), mais aussi que la lumière est la référence
absolue de communication.

Ensemble, ces deux postulats permettent de mettre en évidence que pour relier
deux événements dans des référentiels inertiels différents, une simple
translation\footnote{C'est-à-dire l'ajout d'une distance à une autre.} ne suffit
plus. En effet, en mécanique classique si une personne sur un train très rapide
allume une lampe vers l'arrière du train, une personne sur le quai dudit train
estimerait à la baisse la vitesse de cette lumière (elle serait même nulle si le
train allait à la vitesse de la lumière), par additivité des vitesses. Ceci
n'est pas possible en relativité restreinte, et les transformations amenant d'un
référentiel à l'autre ne sont pas linéaires avec la vitesse\footnote{Nous les
    appelons les \textit{transformations de \textsc{Lorentz}}, du nom du
    mathématicien à l'origine de leur définition. Ça n'est en effet pas
    \textsc{Einstein} le premier à avoir, par exemple, postulé que $E = mc^2$~:
    c'est le premier à avoir regroupé les principes de la relativité en un
concept simple.}. La relativité restreinte nous indiquerait qu'en réalité, du
point de vue du quai le temps dans le wagon du train s'écoule plus
lentement\footnote{La démonstration est laissée à læ lectaire investix…},
une aberration en mécanique classique. Si ceci n'avait jamais été postulé,
c'est que les effets sont minimes aux vitesses que nous connaissons, et les
variations bien plus faibles que ce que nos instruments pouvaient mesurer
alors\footnote{Il faudrait aller à 85\% de la vitesse de la lumière, soit
    $\approx \SI{260000}{km.s^{-1}}$, pour
dilater le temps d'un facteur 2.}. Parmi les conséquences peu intuitives en
plus de la dilatation des durées, nous pouvons citer la relativité de la
simultanéité\footnote{Deux événements à deux endroits différents qui sont
    simultanés dans un référentiel ne le sont en général pas dans un autre
référentiel en mouvement par rapport au premier.} et la contraction des
longueurs.

Avec la relativité générale, le premier principe cité s'élargit et un nouveau
principe, dit d'«~équivalence~», est introduit~:
\begin{enumerate}
    \item les lois de la physique (gravité comprise) sont identiques dans
        \textit{tous} les référentiels~;
    \item la force gravitationnelle sur un corps est équivalente à une
        accélération du référentiel associé audit corps.
\end{enumerate}
Celui-ci est aussi parfois exprimé sous la forme «~les masses graves et inertes
sont égales~»~: nous pouvons en effet définir la masse comme le facteur à
l'origine de l'attraction gravitationnelle (masse grave) mais aussi comme le
facteur traduisant l'inertie d'un système (masse inertielle), c'est-à-dire sa
facilité à changer de vitesse. Un référentiel en chute libre dans un champ de
gravitation est alors un référentiel inertiel (où la relativité restreinte
s'applique).

Ici aussi, si ces idées n'ont pas l'air révolutionnaires, leurs implications
sont nombreuses. Notamment, pour correspondre à la relativité restreinte qui
reste valable dans ce cadre, la mesure des longueurs dans un référentiel
accéléré (non galiléen) ne peut se définir dans un espace plat~: comme une bille
suit la courbe d'une cuvette en y tombant par l'effet de la gravité, la force
d'attraction gravitationnelle est le résultat d'une déformation même de
l'espace-temps forçant des objets massifs à se rapprocher. Ceci constitue une
autre révolution par rapport à la vision newtonienne de la force de gravité,
agissant avec une vitesse jusque-là considérée comme infinie et sans cause bien
définie. Dans ce cadre, les effets gravitationnels aussi mettent du temps à
parcourir l'espace, comme un tissu tendu déformé par une bille prend un certain
temps à reprendre sa forme quand celle-ci est enlevée.

Nous décrivons maintenant les grandeurs et équations décrivant la géométrie de
l'espace-temps en fonction de l'énergie qui y est appliquée.

\subsection{Métrique FLRW en univers plat}\label{ssec:cons}

Le formalisme mathématique nécessaire pour décrire des espaces courbés en
quatre dimensions utilise des objets définis par la géométrie différentielle,
notamment les tenseurs. Ce sont des augmentations des matrices de rotation et
translation en 3D, qui associent à une grandeur sur une dimension une
combinaison des autres. Dans la pratique classique de la physique, nous
définissons une distance $\d L$ entre deux points $u=(x,y,z)$ et $v=(x',y',z')$
d'un espace orthonormé qui vérifie\footnote{C'est le théorème de
\textsc{Pythagore} en 3D.}
\begin{equation}
    (\d L)^2 = (\d x)^2 + (\d y)^2 + (\d z)^2
\end{equation}
avec $\d p = p'-p$ pour $p = x,$ $y$ ou $z$, et nous pouvons définir de manière
indépendante une durée $\d t$ entre deux événements survenant à $t$ et $t'$
telle que $\d t = t'-t$. Nous écrivons de manière compacte pour la première
équation~:
\begin{equation}
    (\d L)^2 = \d w^T\left(\begin{array}{c c c}
            1 & 0 & 0\\
            0 & 1 & 0\\
            0 & 0 & 1\\
    \end{array}\right)\d w
\end{equation}
avec $\d w = v-u$ et $\d w^T$ sa transposée. Cette matrice diagonale ne mélange
pas les dimensions entre elles, et est souvent omise puisqu'elle décrit un effet
nul sur les coordonnées de l'espace, sans rotation ou translation~: une
modification de cette matrice traduirait une modification de la géométrie entre
ces deux points. Nous pourrions comparer cela à un planisphère, qui indique
l'échelle à utiliser pour faire correspondre une distance du plan à une distance
réelle.

Dans un espace-temps plat à quatre dimensions, la «~distance~» entre deux
événements caractérisés par $u=(t,x,y,z)$ et $v=(t',x',y',z')$ est notée $\d s$
et vérifie
\begin{equation}
    (\d s)^2 = -(c\,\d t)^2 + (\d x)^2 + (\d y)^2 + (\d z)^2
\end{equation}
avec $c$ la vitesse de la lumière dans le vide. Cette construction a comme
particularité d'être nulle pour deux points sur le trajet d'un rayon
lumineux\footnote{En effet, le temps est dilaté à l'infini pour un corps
voyageant à la vitesse de la lumière.}.

En notant $x^\mu$ les coordonnées d'un événement quadri-dimensionnel avec $\mu$
allant de 0 à 3, tel que $x^0 = ct$, $x^1 = x$ etc., cette égalité peut
également s'écrire de manière compacte, sous la forme
\begin{equation}
    (\d s)^2 = g_{\mu\nu}\d x^\mu \d x^\nu
\end{equation}
C'est cette grandeur $g_{\mu\nu}$ que nous appelons «~métrique~», et qui définit
la géométrie, courbe ou non, de notre espace. Dans le cas d'un espace plat, nous
voyons qu'elle se définit comme
\begin{equation}
    g_{\mu\nu} = \left(\begin{array}{c c c c}
            -1 & 0 & 0 & 0\\
            0 & 1 & 0 & 0\\
            0 & 0 & 1 & 0\\
            0 & 0 & 0 & 1\\
    \end{array}\right)
\end{equation}
et le chemin le plus court entre deux points est alors défini par une ligne
droite. Une métrique différente de celle-ci décrit un espace dans lequel le
chemin le plus court entre deux points est défini par une courbe dite
«~géodésique~». Assez intuitivement, même si c'est en quatre dimensions, plus la
métrique s'écarte de cette diagonale $(-1,+1,+1,+1)$ plus la géodésique sera
courbée.

Il faut cependant distinguer la courbure de l'Univers-même et la courbure
induite à l'espace-temps par la matière, par exemple. Dans notre cas, nous
supposons un Univers plat (ce qui semble être le cas d'après les mesures), mais
lui laissons la possibilité de varier en taille. S'il n'est pas possible de la
mesurer de manière absolue, il nous suffit de prendre une échelle de distance
entre deux objets aujourd'hui à $t = t_0$ et d'en regarder l'évolution avec le
temps cosmique~; c'est ainsi que nous définissons le facteur d'échelle $a(t)$
tel que $a(t=t_0) = 1$, et de cette manière la métrique de cet Univers devient
celle de \textsc{Friedmann}-\textsc{Lemaître}-\textsc{Robertson}-\textsc{Walker}
(FLRW) dans le cas spécifique d'un Univers plat, c'est-à-dire
\begin{equation}
    g_{\mu\nu} = \left(\begin{array}{c c c c}
            -1 & 0 & 0 & 0\\
            0 & a^2 & 0 & 0\\
            0 & 0 & a^2 & 0\\
            0 & 0 & 0 & a^2\\
    \end{array}\right)
\end{equation}

L'introduction de ce facteur d'échelle permet de définir le taux d'expansion de
l'Univers, que nous appelons paramètre de \textsc{Hubble} $H(t)$, tel que
\begin{equation}\label{eq:hz}
    H(t) = \frac{\dot{a}}{a}\quad\text{avec}\quad\dot{a} = \dv{a}{t}
\end{equation}
et nous appelons constante de \textsc{Hubble} sa valeur aujourd'hui~; nous
notons $H(t=t_0) = H_0$. Notons également qu'une contraction ou dilatation de
l'espace-temps implique inévitablement la variation de la longueur d'onde d'un
photon avec le facteur d'échelle~; nous l'appelons
\textit{redshift}\footnote{Nous décidons de garder le terme dans cette langue
dans la suite pour éviter la verbosité de son équivalent français} pour
«~décalage vers le rouge~», et nous le notons $z$ tel que
\begin{equation}\label{eq:z}
    \frac{a_0}{a(t)} = \frac{\lambda_0}{\lambda_e} = 1+z
\end{equation}
avec $\lambda_e$ la longueur d'onde au moment de l'émission et $\lambda_0$ celle
mesurée à la réception.

\subsection{Équations phares et paramètres cosmologiques}\label{ssec:rgeq}

Si $g_{\mu\nu}$ est une grandeur mathématique utile, la réelle courbure physique
de l'espace-temps est définie par le tenseur d'\textsc{Einstein}, $G_{\mu\nu}$,
construit à partir de $g_{\mu\nu}$. C'est une grandeur physique, continue et
dérivable deux fois avec une divergence nulle\footnote{Autrement dit, qui
conserve le volume.}, qui est égale à 0 quand la courbure est nulle (redonnant
les équations de \textsc{Newton} aux limites classiques). Dans le formalisme de
la relativité générale, la courbure de l'espace-temps en un point est reliée à
l'énergie en ce point~; notamment, la masse (que nous pouvons définir en
énergie) courbe l'espace pour créer l'effet de gravitation. Cette énergie est
définie par un autre tenseur $T_{\mu\nu}$, appelé «~énergie-impulsion~»,
également une grandeur physique de divergence nulle. Ces grandeurs sont alors
reliées par l'équation d'\textsc{Einstein}~:
\begin{equation}\label{eq:ein}
    G_{\mu\nu} = \frac{8\pi\mathcal{G}}{c^4}T_{\mu\nu}
\end{equation}
avec $\mathcal{G}$ la constante gravitationnelle de \textsc{Newton}. Cette
équation est sans doute l'une des équations les plus emblématiques et
fondamentales de la physique moderne, d'une élégance presque inégalée.

Cependant, celle-ci amènera \textsc{Einstein} à se rendre compte qu'un
Univers composé uniquement de matière et de rayonnement ne pouvait donner un
Univers statique~: le facteur d'échelle $a$ était destiné à varier, et donc
l'Univers à avoir potentiellement un début et une fin. Cette découverte était
inconcevable dans sa conception du monde, et pour contrer ce phénomène il
modifia son équation~\ref{eq:ein} en y ajoutant une constante cosmologique
$\Lambda$ qui peut s'introduire d'un côté ou de l'autre du signe égal, tel que
\begin{equation}\label{eq:einlamb}
    G_{\mu\nu} + \Lambda g_{\mu\nu} = \frac{8\pi\Gc}{c^4}T_{\mu\nu}
    \Longleftrightarrow
    G_{\mu\nu} = \frac{8\pi\Gc}{c^4}T_{\mu\nu} - \Lambda g_{\mu\nu} 
\end{equation}

Si d'une manière purement mathématique les deux équations sont équivalentes,
leurs motivations physiques sont différentes~: dans le premier cas, cela revient
à modifier intrinsèquement les lois de la gravitation en y introduisant un
phénomène complètement décorrélé de l'aspect géométrique de sa manifestation~;
dans le second, ce terme serait une nouvelle source de champ gravitationnel, une
forme d'énergie ne se diluant pas et donc indépendante du temps. Le fait que ce
terme puisse être des deux côtés du signe «~=~» signifie que nous ne pouvons pas
distinguer son origine. S'il est revenu sur sa décision plus tard, son inclusion
reste toute fois intéressante et nous conservons ce terme quitte à le trouver
nul par la suite.

Pour décrire la composition de l'Univers en énergie \textit{via} $T_{\mu\nu}$,
nous utilisons le modèle du fluide parfait. Il est défini par deux quantités~:
$\rho$, sa masse volumique (ou d'une manière plus générale, sa densité
d'énergie)~; et $p$, sa pression hydrostatique (c'est-à-dire une force par unité
de surface sur des parois, imaginaires ici). L'isotropie de notre Univers
implique que ce tenseur est diagonal dans son référentiel de repos puisqu'il
n'y a pas de direction privilégiée à sa pression, et que les trois composantes
d'espace ont la même valeur $p/c^2$. Nous pouvons donc l'écrire
\begin{equation}
    T_{\mu\nu} = \left(\begin{array}{c c c c}
            \rho & 0 & 0 & 0\\
            0 & p/c^2 & 0 & 0\\
            0 & 0 & p/c^2 & 0\\
            0 & 0 & 0 & p/c^2\\
    \end{array}\right)
\end{equation}

Le fait que l'énergie soit conservée dans le temps se traduit par la divergence
nulle du tenseur énergie-impulsion, de telle sorte qu'un fluide parfait est
caractérisé par l'équation de conservation
\begin{equation}\label{eq:cons}
    \dot{\rho} = -\frac{3H}{c^2}\rho(1+w)
\end{equation}
avec $w \triangleq p/\rho c^2$\footnote{$\triangleq$ signifie «~égal par
définition~».} le paramètre d'état du fluide en question. En effet, après
résolution de cette équation différentielle nous déduisons
\begin{equation}\label{eq:rho}
    \rho(t) \propto a(t)^{-3(1+w)}
\end{equation}
et le fluide présentera des comportements variés selon la valeur de $w$~:
\begin{itemize}
    \item si $w=0$, c'est-à-dire $p=0$, nous aurons $\rho \triangleq \rho_M
        \propto a^-3$~: c'est le comportement typique de la matière
        non-relativiste
        qui se déplace à faible vitesse et a donc une pression négligeable tout
        en ayant une densité d'énergie non-nulle se diluant avec le volume~;

    \item si $w=1/3$, nous aurons $\rho \triangleq \rho_R \propto a^{-4}$~:
        c'est le comportement de la limite ultra-relativiste ($v \approx c$),
        propre à l'énergie de radiation comme celle des photons ou des
        neutrinos. Nous la décrivons comme un fluide dont la densité se dilue
        comme la matière non-relativiste (avec $\rho_M \propto a^{-3}$) mais
        dont la longueur d'onde s'étire également avec le facteur d'échelle,
        diminuant la densité d'énergie par un facteur $a^{-1}$ supplémentaire~;

    \item si $w=-1$, c'est-à-dire $p < 0$, nous aurions alors $\rho \propto a^0
        = \text{constante}$. Ce paramètre d'état décrirait alors un fluide à
        pression négative mais dont la densité d'énergie ne se dilue pas avec le
        facteur d'échelle… autrement dit, un comportement qui pourrait convenir
        pour décrire l'effet énergétique de la constante cosmologique. Nous
        qualifions une telle forme d'énergie de «~sombre~».
\end{itemize}

À partir de cette idée de fluide parfait et avec l'équation~\ref{eq:einlamb}
dans le cadre de la métrique FLRW dans un Univers plat, nous pouvons dériver une
autre équation fondamentale de la relativité générale, à savoir
\begin{equation}\label{eq:fried}
    H^2 \triangleq \left( \frac{\dot{a}}{a} \right)^2 =
    \frac{8\pi\mathcal{G}\rho}{3c^2} + \frac{\Lambda c^2}{3}
\end{equation}
que nous appelons équation de \textsc{Friedmann} \citep{friedmann1922}.

Pour simplifier cette forme, nous pouvons prendre sa valeur à $t = t_0$ pour
définir une densité dite «~critique~»~:
\begin{equation}
    \rho_c = \frac{3c^2H_0{}^2}{8\pi\mathcal{G}}
\end{equation}
et ainsi, selon la nature des fluides composant l'Univers, nous séparons les
différents types de densités $\rho$ pour obtenir la forme~:
\begin{equation}
    H^2 \triangleq \left( \frac{\dot{a}}{a} \right)^2 =
    H_0{}^2 \left[ \frac{\rho_R(t_0)}{\rho_c}\frac{\rho_R}{\rho_R(t_0)} +
                   \frac{\rho_M(t_0)}{\rho_c}\frac{\rho_M}{\rho_M(t_0)} +
                   \frac{\Lambda c^2}{3 H_0{}^2}
               \right]
\end{equation}

Comme $\rho_R \propto a^{-4}$, $\DS\frac{\rho_R}{\rho_R(t_0)} =
\left(\frac{a_0}{a}\right)^4$, et de même $\DS\frac{\rho_M}{\rho_M(t_0)} =
\left(\frac{a_0}{a}^3\right)$~;
nous pouvons donc réécrire
\begin{equation}
    H^2 \triangleq \left( \frac{\dot{a}}{a} \right)^2 =
    H_0{}^2 \left[ \Omega_R \left( \frac{a_0}{a} \right)^4 +
        \Omega_M \left( \frac{a_0}{a} \right)^3 +
    \Omega_\Lambda \right]
\end{equation}
et finalement obtenir, avec l'équation~\ref{eq:z}~:
\begin{equation}\label{eq:h2}
    H^2 \triangleq \left( \frac{\dot{a}}{a} \right)^2 =
    H_0{}^2 \left[ \Omega_R \left( 1+z \right)^4 +
        \Omega_M \left( 1+z \right)^3 +
    \Omega_\Lambda \right]
\end{equation}
en notant $\Omega_X \triangleq \DS \frac{\rho_X(t_0)}{\rho_c}$ pour $X = R$ ou
$M$, et $\Omega_\Lambda = \DS \frac{\Lambda c^2}{3 H_0{}^2}$. Ce sont ces
densités réduites, dont la somme est égale à 1 par construction (autrement dit,
$\sum_i \Omega_i = 1$) que nous appelons «~paramètres cosmologiques~». Ils
constituent avec $H_0$ la base de l'histoire et de l'évolution de l'Univers,
étant donné qu'ils caractérisent la variation du taux d'expansion (au carré).
Nous présentons Figure~\ref{fig:aevol} différentes évolutions du facteur
d'échelle selon la répartition énergétique de l'Univers avec $H_0 =
\SI{70}{km.s^{-1}.Mpc^{-1}}$\footnote{Cette unité, qui peut paraître
    déconcertante, est bien homogène à des \si{s^{-1}} puisque nous parlons de
    \textit{taux} d'expansion, mais elle est plus appréciable sous cette forme
    puisque nous pouvons y lire que pour chaque \si{Mpc} (unité de distance
    astronomique) qui nous sépare actuellement d'un objet, l'expansion de
    l'Univers le fait s'écarter de nous de $\approx \SI{70}{km.s^{-1}}$, ou
\SI{252000}{km.h^{-1}}.}.

\begin{figure}[ht]
    \centering
    \begin{subfigure}[c]{.48\linewidth}
        \centering
        \includegraphics[width=\linewidth]{aevol_o}
        \label{fig:aevol_l}
    \end{subfigure}
    \hfill
    \begin{subfigure}[c]{.48\linewidth}
        \centering
        \includegraphics[width=\linewidth]{Hevol}
        \label{fig:aevol_l}
    \end{subfigure}
    \caption[Évolution du facteur d'échelle en fonction de la répartition
    énergétique de l'Univers et de $H_0$]{Évolution du facteur d'échelle en
        fonction de la répartition énergétique de l'Univers et de $H_0$. Le
        temps est indiqué par rapport à aujourd'hui (facteur d'échelle = 1).
        \textit{À gauche}~: variation de $\Omega_\Lambda$ et $\Omega_M$ pour
        $H_0 = \SI{70}{km.s^{-1}.Mpc^{-1}}$. Les premiers modèles impliquent un
        Univers en expansion éternelle, le dernier un Univers qui s'effondrera
        sous l'effet de la masse. \textit{À droite}~: variation de $H_0$ pour
        $\Omega_M = \num{0.315}$ et $\Omega_{\Lambda} = \num{0.685}$. Plus $H_0$
        est élevé, plus l'évolution de l'Univers est rapide. La valeur de
        \SI{500}{km.s^{-1}.Mpc^{-1}} trouvée par \cite{hubble1929} impliquerait
        que l'Univers serait plus jeunes que le système Solaire
    (\SI{4.5}{Gans}).}\label{fig:aevol}
\end{figure}

\subsection{Modèle standard de la cosmologie~: \lcdm}\label{sec:MS}

Si jusque-là notre approche était générale et historique (sauf pour l'Univers
plat), nous effectuons un saut dans le futur pour discuter de la cosmologie
moderne. L'origine éventuelle et le futur de l'Univers sont en effet régis par les
valeurs de ces paramètres cosmologiques, et si \textsc{Einstein} n'eût pas la
chance de vivre jusqu'à la fin du XX\ieme~siècle, nous avons des valeurs de ces
paramètres. Notamment, \cite{riess1998} et~\cite{perlmutter1999} ont mesuré,
grâce à des SNe~Ia, les valeurs des paramètres cosmologiques $\Omega_i$. Leurs
résultats indiquent que l'énergie totale de l'Univers serait à 70\% de l'énergie
sombre, autrement dit une forme d'énergie qui ne nous est pas tangible et dont
nous avons seulement pu caractériser les effets à l'échelle de l'Univers entier~; le
reste de cette énergie est alors principalement de la matière, qui composerait
30\% de l'énergie de l'Univers. La Figure~\ref{fig:snlambda} présente la
contrainte apportée par~\cite{riess1998} sur les paramètres $\Omega_\Lambda$ et
$\Omega_M$ ainsi que les contraintes actuelles que les SNe~Ia permettent (nous
détaillons le principe de ces mesures dans la section suivante).

\sidecaptionvpos{figure}{c}
\begin{SCfigure}[1][ht]
    \centering
    \includegraphics[width=.6\linewidth]{scolnic_snlambda}
    \caption[Contraintes sur les paramètres cosmologiques $\Omega_\Lambda$ et
    $\Omega_M$ par les SNe~Ia seulement]{Contraintes sur les paramètres
        cosmologiques $\Omega_\Lambda$ et $\Omega_M$ par les SNe~Ia seulement,
        mettant en évidence l'existence de l'énergie sombre. Les contours de
        confiance à 68 et 95\% sur les paramètres sont montrés pour les mesures
        de~\cite{riess1998} (R98 Discovery Sample) et celles de l'échantillon
        Pantheon \textit{en rouge}. Figure
    de~\cite{scolnic2018}.}\label{fig:snlambda}
\end{SCfigure}

Cette mesure implique la découverte de l'expansion accélérée de l'Univers, étant
donné qu'aucune autre forme d'énergie ne rivalise actuellement avec celle-ci et
que, contrairement aux autres, sa densité ne se dilue pas avec le temps~;
autrement dit, l'Univers est voué, sauf preuve du contraire, à s'étendre
indéfiniment en isolant les structures qui ne sont pas en interaction
gravitationnelle suffisamment forte avec les autres. Cette découverte a mené
Saul \textsc{Perlmutter}, Adam \textsc{Riess} et Brian \textsc{Schmidt} à
obtenir le prix \textsc{Nobel} en 2011, après confirmation de ces mesures par
des sondes indépendantes permettant de mieux contraindre les paramètres
estimés\footnote{En effet, la mesure d'une distance avec une règle et avec le
    temps de vol d'un laser par exemple sont des méthodes de mesure qui
    permettent une estimation de la valeur attendue, mais reposent toutes les
    deux sur des principes physiques différents qui impliquent des dépendances
    et des erreurs qui se combinent pour mieux déterminer la longueur en
question.}.

Le modèle vers lequel ces différentes sondes convergent est appelé le «~modèle
de concordance~». Sans les détailler, en dehors des SNe~Ia il existe des mesures
de ces paramètres cosmologiques \textit{via} le fonds diffus cosmologique
(\textit{Cosmic Microwave Background}, CMB) et les oscillations acoustiques des
baryons (\textit{Baryon Acoustic Oscillations}, BAO) notamment. Ce modèle, dit
«~standard~», décrit un Univers plat constitué de matière
sombre\footnote{C'est-à-dire une forme d'énergie similaire à la matière, avec un
    comportement gravitationnel attractif et un paramètre d'état $w=0$, mais
    invisible par rayonnement électromagnétique et n'interagissant pas avec la
matière ordinaire non plus.} froide\footnote{Nous entendons par là
    non-relativiste, c'est-à-dire se déplaçant à des vitesses faibles devant
celle de la lumière.} et d'une constante cosmologique $\Lambda$ (probablement
sous forme d'énergie sombre)~; il est pour cela appelé $\Lambda CDM$ pour
\textit{Lambda Cold Dark Matter}. Il permet de rendre compte de l'origine et
de la structure du fonds diffus cosmologique, de la composition en atomes et
de la structure des grandes échelles de l'Univers, ainsi que de son
expansion accélérée. Si ses capacités prédictives sont remarquables, la
nature précise de la matière et de l'énergie sombres reste pour le moment un
mystère. \cite{planck2018} rapportent trouver $\Omega_M = \num{0.3111} \pm
\num{0.0056}$ et $\Omega_\Lambda = \num{0.6886} \pm \num{0.0056}$, avec
$\Omega_R$ considéré comme nul étant donné sa dépendance en $a^{-4}$.

Il existe d'autres modèles de cosmologie, laissant certains paramètres varier~;
c'est le cas du modèle $w$CDM pour lequel le paramètre d'état de l'énergie
sombre n'est pas fixé à \num{-1}~: dans ce cas, nous écrivons
l'équation~\ref{eq:h2} dans sa forme générale~:
\begin{equation}\label{eq:h2w}
    H^2 \triangleq \left( \frac{\dot{a}}{a} \right)^2 =
    H_0{}^2 \left[ \Omega_R \left( 1+z \right)^4 +
        \Omega_M \left( 1+z \right)^3 +
    \Omega_\Lambda \left( 1+z \right)^{3(1+w)} \right]
\end{equation}
où $w=-1$ correspond à une constante cosmologique ne se diluant pas avec le
temps. Nous ne parvenons cependant pas encore à réfuter l'un ou l'autre des
modèles avec les mesures actuelles des paramètres~: la combinaison SN+CMB+BAO
de~\cite{scolnic2018} trouvent en effet $w = \num{-1.014}\pm\num{0.040}$,
et~\cite{planck2018} trouvent $w = \num{-1.03}\pm\num{0.03}$, valeurs
compatibles avec $\Lambda$CDM. Nous présentons Figure~\ref{fig:cosmocomb} les
contraintes combinées de ces sondes pour $w$CDM.

\begin{SCfigure}[1][ht]
    \centering
    \includegraphics[width=.6\linewidth]{scolnic_combined}
    \caption[Contraintes sur les paramètres cosmologiques $w$ et $\Omega_M$ par
    la combinaison SNe~Ia, CMB et BAO]{Contraintes à 68 et 95\% sur les
        paramètres cosmologiques $w$ et $\Omega_M$ par la combinaison SNe~Ia
        (\textit{en rouge}), par le CMB (\textit{en bleu}) fournies par la
        collaboration~\cite{planck2015}. Les contours \textit{jaunes} combinent
        le CMB et le BAO~\citep{alam2015}~; les contours \textit{noirs} le CMB
    et les SNe~Ia. Figure de~\cite{scolnic2018}.}\label{fig:cosmocomb}
\end{SCfigure}
\sidecaptionvpos{figure}{t}

Il est cependant sujet à une tension historique du fait de l'incompatibilité de
la mesure de $H_0$ entre le CMB et les SNe~Ia~: en effet, ces deux méthodes
trouvent des valeurs respectives de \SI{67.4\pm0.5}{km.s^{-1}.Mpc^{-1}} et
\SI{73.04\pm1.04}{km.s^{-1}.Mpc^{-1}}. L'étude de cette incohérence est
actuellement au cœur de la cosmologie moderne, chacune de ces sondes ayant fait
ses preuves quant à la fiabilité de leurs mesures.

\section{Mesures cosmologiques}\label{sec:dist}

Nous discutons dans cette section des grandeurs qui servent un intérêt à la
compréhension de l'impact des paramètres cosmologiques sur des grandeurs
appréciables ou pour notre étude.

\subsection{Âge de l'Univers}\label{ssec:age}

Au cours du XX\ieme~siècle, l'expansion avérée de l'Univers a amené la
communauté scientifique à supposer qu'il devait exister, à une époque lointaine,
un état de l'Univers où son facteur d'échelle était infiniment proche de 0, soit
infiniment petit et donc infiniment chaud en concentrant toute l'énergie.
D'abord appelé de manière dérisoire \textit{Big Bang} en 1949 par un
astrophysicien qui contestait cette idée, celle-ci a été confortée avec les
mesures qui ont suivi. Pour l'estimer, il nous faut faire correspondre le
facteur d'échelle avec le temps.

Nous utilisons pour cela le redshift $z$ d'un objet, qui est caractérisé par
l'équation~\ref{eq:z}, et la définition du taux d'expansion $H(z)$ de
l'équation~\ref{eq:hz}. Nous définissons alors l'âge de l'Univers comme celui
que nous trouverions pour un corps de redshift infini. Nous pouvons le relier au
temps qui s'est écoulé entre son émission et sa réception par~:
\begin{align}
    t(z) & = \int_{0}^{t(z=\infty)} \d t' = \int_{0}^{a(z=\infty)} \frac{\d a'}{\dot{a}'}
    \nonumber\\
         & = \int_{0}^{a(z=\infty)} \frac{\d a'}{a'H(a')} = \int_{0}^{\infty} \frac{\d
         z'}{(1+z')H(z')}\nonumber\\
         & = \frac{1}{H_0} \int_{0}^{\infty} \frac{\d z'}{(1+z')E(z')}
\end{align}
où $E(z)$ se déduit de la définition de $H^2$ de l'équation~\ref{eq:h2}~:
\begin{equation}\label{eq:ez}
    E(z) \triangleq \frac{H(z)}{H_0} = \left[ \Omega_R(1+z)^4 + \Omega_M(1+z)^3
    + \Omega_\Lambda\right]^{1/2}
\end{equation}
L'âge de l'Univers dépend donc de la répartition de son énergie selon les
différentes formes qui le composent, c'est-à-dire du modèle. Dans le cadre du
modèle $\Lambda$CDM, il est estimé à $t_0 = \SI{13.797\pm0.023}{Gans}$
par~\cite{planck2018}. Une modification de la fraction ou du paramètre d'état de
l'énergie sombre, notamment, amènerait à une variation de cette valeur.

\subsection{Distance de luminosité}\label{ssec:dl}

Si une variation de l'âge de l'Univers n'aurait pas grande incidence sur notre
rapport au monde, un effet notable de l'histoire de son expansion se remarque
sur les distances des sources lumineuses avec nous. En effet, une source
lumineuse dans l'espace diffuse sa luminosité\footnote{C'est l'énergie émise par
unité de temps.} $L$ sur une sphère centrée autour de son point d'émission~:
pour an observataire à une distance $d$ de cette source, la sphère est de
surface $4\pi d^2$ et le flux\footnote{C'est-à-dire l'énergie par unité de
surface, c'est une fraction de l'énergie totale.} $F$ de l'objet se calcule
selon~:
\begin{equation}\label{eq:f}
    F = \frac{L}{4\pi d^2}
\end{equation}

Comme nous l'avons vu, la distance entre deux objets dépend non seulement de
leurs mouvements propres respectifs, mais également de l'étirement de
l'espace-temps qui a pu s'y ajouter entre l'émission et la réception, qui domine
assez rapidement sur le mouvement propre des corps. En astrophysique, elle se
mesure parsecs, de symbole pc. À titre de comparaison, la distance entre la
Terre et le Soleil est de \SI{5e-6}{pc}\footnote{Autrement écrit~:
\SI{0.000005}{pc}.}~; cette distance est au parsec ce que l'épaisseur typique
d'un cheveu $a$ est au mètre, à savoir $a \approx \SI{5e-6}{m}$\footnote{S'il
    faut donc \num{200000} cheveux \textit{dans leur épaisseur} pour faire une
    distance de \SI{1}{m}, il faudrait \num{200000} fois la distance
Terre-Soleil pour avoir \SI{1}{pc}.}.

En négligeant donc les vitesses particulières et avec un raisonnement similaire
à celui pour déterminer l'âge de l'Univers de la section~\ref{ssec:age} (nous
déterminons cette distance dans un espace en expansion en calculant
$\int_{t(z)}^{t_0} \frac{c\d t'}{a(t')}$), nous obtenons cette distance dite
«~comobile~» par~:
\begin{equation}
    d_C(z) = \frac{c}{H_0} \int_{0}^{z} \frac{\d z'}{E(z')}
\end{equation}

Cependant, les photons qui traversent l'espace subissent un effet supplémentaire
dû à cette expansion~: en plus de l'allongement de la distance, ils subissent un
effet de dilution énergétique traduit par le décalage vers le rouge de leur
longueur d'onde, c'est-à-dire le redshift~; nous définissons ainsi la distance
de luminosité par~:
\begin{equation}\label{eq:dl}
    d_L(z) = (1+z)d_C(z) = \frac{c(1+z)}{H_0} \int_{0}^{z} \d z'
    \left[
        \Omega_R \left( 1+z \right)^4 +
        \Omega_M \left( 1+z \right)^3 +
        \Omega_\Lambda %\left( 1+z \right)^{3(1+w)}
    \right]^{-1/2}
\end{equation}

C'est cette distance-là qui est utilisée dans l'équation~\ref{eq:f} pour
déterminer le flux d'un astre lumineux. Sa dépendance avec le redshift
implique qu'à bas redshift, nous pouvons prendre au premier ordre en $z$~:
\begin{equation}\label{eq:dlz0}
    d_L \underset{z \ll 1}{=} \frac{zc}{H_0}
\end{equation}

\subsection{Intérêt des supernovae de type Ia}\label{ssec:intsne}

Nous pouvons commencer à entrevoir que la mesure du flux d'un astre pourrait
permettre de contraindre les paramètres cosmologiques $\Omega_i$ de
l'équation~\ref{eq:dl}. Nous utilisons cependant d'autres grandeurs reliées au
flux émis pour exprimer ce que nous mesurons~: les magnitudes. La magnitude
apparente $m$ d'un objet émettant un flux $F$ est définie \textit{via} la
relation de~\cite{pogson1856}~:
\begin{equation}\label{eq:m}
    m = -2,5\log \left(F\right) + \text{cst} =
    -2,5\log \left( \frac{L}{4\pi d^2} \right) + \text{cst}
\end{equation}
Elle s'exprime en magnitudes (de symbole «~mag~») et, par construction, augmente
quand la luminosité diminue\footnote{Le Soleil a une magnitude apparente de
    -\SI{26.74}{mag}~; la pleine Lune de -\SI{12}{mag}~; l'œil humain peut
    percevoir jusqu'à \SI{6}{mag} sans pollution lumineuse~; le télescope
spatiale \textsc{Hubble} a comme limite \SI{32}{mag}.}.

D'une part, $L$ n'est pas connue et mesurable \textit{a priori}, mais d'autre
part cette constante n'est pas définie. Pour s'en restreindre, nous pouvons
utiliser une référence de magnitude connue qui fait intervenir la même constante
et soustraire les deux magnitudes apparentes. Cette référence est souvent celle
de l'étoile Véga, la cinquième étoile la plus brillante du ciel dont nous
connaissons la distance par une mesure directe (la parallaxe, que nous ne
détaillons pas ici) et fut autour de -\SI{12000}{ans} l'étoile
polaire\footnote{Cette définition n'est en effet pas fixe et absolue~; nous
    définissons l'étoile polaire comme une étoile visible à l'œil nu et qui se
    trouve sur l'axe de rotation de la Terre (au-dessus d'un de ses pôles, il
    peut donc y en avoir deux). Comme cet axe tourne dans le temps, comme le
    fait une
toupie, l'étoile considérée comme polaire change.}, mais il reste dans ce cas que
la luminosité $L$ du corps dont nous exprimons $m$ et la luminosité $L_0$ de la
référence sont encore inconnues.

Pour se passer de ce terme dans l'équation~\ref{eq:m}, nous définissons la
magnitude dite «~absolue~», $M$, comme étant la magnitude que nous mesurerions
si l'objet était placé à une distance $d_0 = \SI{10}{pc}$ de nous~: $M =
-2,5\log \left( \frac{L}{4\pi d_0^2} \right) + \text{cst}$. Dans ce cas, la
soustraction donne ce que nous appelons le module de distance $\mu$, tel que~:
\begin{align}\label{eq:mucosmo}
    \mu = m - M = -2,5\log \left( \frac{L}{4\pi d^2} \right) & +
    \cancel{\text{cst}} + 2,5\log \left( \frac{L}{4\pi d_0^2} \right) -
    \cancel{\text{cst}}\nonumber\\
    \mu = m-M & = 5\log \left( \frac{d_L}{d_0} \right)
\end{align}
en utilisant les propriétés de la fonction $\log$.

Ainsi, à partir de la mesure de $m$ et de la connaissance de $M$, nous pouvons y
faire correspondre la valeur de droite de l'équation~\ref{eq:mucosmo} qui est
reliée aux paramètres cosmologiques par l'équation~\ref{eq:dl}. C'est de cette
manière que~\cite{hubble1929} a rempli son diagramme (dit «~diagramme de
\textsc{Hubble}~») module de distance-redshift et déterminé la première valeur
de $H_0$, qu'il avait estimée à \SI{500}{km.s^{-1}.Mpc^{-1}} (soit 7 fois
supérieure à la valeur estimée aujourd'hui).

Nous remarquons cependant que cette magnitude absolue n'est pas une donnée, et
sans sa connaissance il est impossible de déterminer la distance d'un
corps\footnote{Prenons deux bougies qui éclairent différemment~: si vous ne
    voyiez que deux points lumineux dont l'un est plus fort que l'autre, comment
savoir laquelle des bougies est la plus proche~?}. Un type d'objet pour lequel
$M$ est connu et constant dans sa catégorie est qualifié de «~chandelle
standard~».

C'est notamment pour cela que les SNe~Ia se révèlent d'une importance capitale
dans cette étude. Nous pouvons distinguer trois raisons majeures pour cela~:
\begin{enumerate}
    \item ces astres sont connus pour leur régularité dans leur luminosité
        totale émise, et donc dans leur magnitude absolue. Ce sont ainsi des
        objets permettant cette calibration~;

    \item ce sont des événements cosmologiques d'une luminosité très importante,
        visibles à des distances (ou des redshifts) relativement grandes ($z
        \approx 1.2\footnote{C'est-à-dire des objets qui ont émis leur
        lumière quand l'Univers était approximativement moitié plus jeune.}$ avec
        des télescopes sur Terre). C'est important puisque plus nous avons des
        points de mesure variés sur la distance $d_L$, plus nous aurons de
        contraintes sur les paramètres cosmologiques. Notamment, pour $z \ll
        1$ l'équation~\ref{eq:dlz0} implique que leur module de distance ne
        dépend que de $H_0$, ce qui permet de contraindre ce paramètre
        indépendamment des autres~;

    \item enfin, étant des événements explosifs survenant à la fin de vie de
        certaines étoiles, leur nombre augmente naturellement avec le temps,
        permettant une augmentation continue de possibles points de mesure.
\end{enumerate}

Ces caractéristiques sont à l'origine de leur utilisation
par~\cite{perlmutter1999}, permettant de déterminer que l'Univers est composé à
$\approx$ 70\% d'énergie sombre (voir Figure~\ref{fig:perlm_w}).

\begin{figure}[]
    \centering
    \includegraphics[width=\linewidth]{perlmutter_sne-w}
    \caption[Diagramme de \textsc{Hubble} avec les 42 SNe~Ia historiques de
    1999]{(a)~: diagramme de \textsc{Hubble} avec les 42 SNe~Ia historiques
        utilisées par~\cite{perlmutter1999} d'où provient la figure. (b)~:
        résidus de \textsc{Hubble} (différence de la magnitude mesurée avec le
        meilleur modèle ajusté). (c)~: écart à la valeur attendue du modèle en
    prenant en compte les erreurs de mesure.}\label{fig:perlm_w}
\end{figure}

Nous discutons dans le chapitre suivant des définitions et caractéristiques de
ces astres afin d'en cerner les avantages d'une part mais également les limites
d'autre part.

\newpage

\thispagestyle{plain}
% \vspace*{-3cm}
\vfill
\minilof
\vfill
\minilot
\vfill

% \bibliographystyle{../main/aa_url}
% \shorthandoff{:}
% \bibliography{../chapters/99_references}

\end{document}
