\documentclass[../main/main.tex]{subfiles}
\begin{document}
\mainmatter

% \dominitoc
% \faketableofcontents
% \dominilof
% \fakelistoffigures
% \dominilot
% \fakelistoftables

\chapter{Contexte cosmologique}\label{ch:cosmo}

\epigraph{\openquote\textit{Time is an illusion, a construct made out of human
memory}\closequote}{Blake \textsc{Crouch}, \textit{Revelations}}

Bien que la cosmologie ne s'en tienne pas aux concepts récents tels qu'on les
connaît et les vulgarise, c'est avec les travaux d'\textsc{Einstein} au début du
XX\ieme~siècle que notre compréhension du monde cosmique prend son essor. 

\vfill
\minitoc
\vfill

\newpage

\section{Bases de relativité générale}\label{sec:11}
\subsection{Concepts initiaux}\label{ssec:RG}
\subsection{Métrique et équations de conservation}\label{ssec:112}
\subsection{Définition de la constante cosmologique}\label{ssec:lambda}

\section{Introduction du modèle standard de la cosmologie}\label{sec:MS}
\subsection{Univers plat, homogène et isotrope}\label{ssec:plat}
\paragraph*{Univers plat}
\subsection{Métrique de Friedmann-Lemaître-Robertson-Walker}\label{ssec:FLRW}
\subsection{Le modèle \lcdm}\label{ssec:LCDM}

\section{Mesure cosmologiques}\label{sec:dist}
\subsection{Âge de l'Univers}\label{ssec:age}
\subsection{Distance de luminosité}\label{ssec:dl}
\subsection{Intérêt des supernovae de type Ia}\label{ssec:intsne}

\newpage

\thispagestyle{plain}
% \vspace*{-3cm}
\vfill
\minilof
\vfill
\minilot
\vfill

% \bibliographystyle{../main/aa_url}
% \shorthandoff{:}
% \bibliography{../chapters/99_references}

\end{document}
