\documentclass[../main/main.tex]{subfiles}
\begin{document}

\chapter*{Introduction g\'en\'erale}
%\appch{Introduction g\'en\'erale}{ch:intro}

L'Univers a toujours fasciné l'esprit, et les tentatives de compréhension et de
description sont nombreuses. En Grèce antique, l'ordre était de mise, et le
philosophe \textsc{Anaximandre} proposait de décrire l'Univers comme une sphère
dont la Terre serait le centre, entourée d'une sphère de poussière et d'une
sphère de feu concentriques~; les étoiles étaient pour lui des trous dans la
première laissant passer le feu de la seconde. En étudiant le système Solaire,
\textsc{Kepler} publie en 1596 son \textit{Mysterium Cosmographicum} dans lequel
il en propose un modèle où les distances des (à l'époque) 6 planètes seraient
décrites par une imbrication des cinq solides de \textsc{Platon}.

Si ces idées peuvent paraître saugrenues aujourd'hui, c'est que d'autres
théories et expériences ont permis de les invalider~; pour autant les idées
apportées au XX\ieme\ siècle par \textsc{Einstein} avec la Relativité Générale
\citep{einstein1915} n'étaient \textit{a priori} pas moins déconcertantes. Par
exemple, en découvrant que sa théorie impliquait que l'Univers n'était pas
statique et pour correspondre à sa vision personnelle, il dut y rajouter un
terme compensant l'effondrement d'un univers qui ne serait composé que de
matière~: la constante cosmologique, étirant l'espace-temps de l'Univers, était
née.

S'il l'a ensuite abandonnée du manque de motivation rigoureuse, les observations
des années 30 par \textsc{Le Maître} et \textsc{Hubble} \citep{hubble1929} ont
montré que toutes les galaxies s'écartent de nous, avec une vitesse croissante
avec la distance \textit{via} la constante de \textsc{Hubble} $H_0$ décrivant le
taux d'expansion actuel de l'Univers. La cosmologie observationnelle moderne en
était alors à ses débuts, permettant d'inclure le paramètre d'état de l'énergie
sombre $w$ dans les équations d'\textsc{Einstein} pour décrire ce phénomène
semblant dilater le tissu de l'espace-temps.

Ces deux termes sont aujourd'hui au cœur de toute la cosmologie, décrivant son
âge, son passé et son futur. À ce titre, les SNe~Ia se sont illustrées comme
étant des outils indispensables à l'étude de ces valeurs. Elles ont en effet,
d'une part, permis à \textsc{Riess}, \textsc{Schmidt} et \textsc{Perlmutter} de
publier deux articles en 1998 et 1999 \citep{riess1998, perlmutter1999},
indiquant que l'énergie sombre constitue approximativement 70\% du budget
énergétique total de l'Univers. Cette mesure implique la découverte de son
expansion accélérée, et a été récompensée d'un prix \textsc{Nobel} en 2011.
D'autre part, ces astres sont utilisés de manière extensive pour la mesure de
$H_0$, remplaçant les galaxies utilisées par \textsc{Hubble} en 1929.

Leur utilisation repose sur le fait que leur luminosité est constante. En effet,
deux étoiles n'émettent pas forcément la même luminosité, ce qui rend
pratiquement impossible par cette seule donnée le fait de savoir laquelle est la
plus proche de nous. Or, les SNe~Ia sont le résultat de la mort d'étoiles
avec un mécanisme particulier qui les fait exploser à une masse standard, et
donc avec une énergie totale similaire, ce qui leur vaut le terme de «~chandelle
standard~». Cette caractéristique permet de facilement déterminer leur
distances.

Seulement, cette hypothèse n'est pas si exacte et la nature précise de leur
mécanisme d'explosion est encore méconnue, ce qui fait qu'il existe une
variabilité dans cette luminosité intrinsèque. Il a en réalité fallu trouver une
corrélation entre cette luminosité et d'autres de leurs propriétés pour
déterminer une relation amenant à la standardisation de la mesure de leur
distance~; on appelle aujourd'hui les SNe~Ia des chandelles
\textit{standardisables}.

C'est dans ce contexte que s'ancre notre thèse. Dans l'optique de continuer à
améliorer cette standardisation, la cosmologie observationnelle utilisant les
SNe~Ia cherche des corrélations supplémentaires nous permettant de les décrire
au mieux, notamment en distinguant deux sous-populations ayant des propriétés
différentes selon la valeur d'un paramètre. Seulement, étant par nature
non-reproductible, cette science se trouve être fastidieuse dans son
développement. Différents paramètres sont proposés comme étant à l'origine des
propriétés physiques des SNe~Ia, amenant à des standardisations différentes et
donc à des mesures des paramètres cosmologiques différents. Si cette
méconnaissance de la physique intrinsèque des SNe~Ia n'était pas dominante avec
peu de statistique de points de mesure, la cosmologie moderne acquiert de plus
en plus de données à un rythme frénétique, passant de $\approx$ 100 données en
1999 à $\approx$ \num{1300} aujourd'hui et bientôt \num{10000} dans les années à
venir.

Notre objectif est donc de contribuer à la compréhension de la physique des
SNe~Ia en étudiant les implications d'une corrélation largement étudiée mais
encore peu implémentée aujourd'hui~: celle des propriétés d'une SN~Ia avec l'âge
de son étoile d'origine.

Nous posons la base de cette étude dans les trois premiers chapitres, à savoir
le contexte cosmologique, le fonctionnement des SNe~Ia et les corrélations
utilisées aujourd'hui qui motivent notre analyse.

Dans un deuxième temps, nous décrirons également en trois chapitres ce qui a
constitué la première partie de ces trois ans de recherche, à savoir la
description des sondages utilisés pour l'établissement d'un échantillon de
données permettant l'analyse des propriétés des SNe~Ia avec leur âge. Dans cette
partie nous décrirons le modèle d'évolution que nous proposons dans les analyses
cosmologiques.

Finalement, nous traiterons dans les chapitres suivants de l'implémentation de
ce modèle dans la chaine d'analyse cosmologique la plus utilisée dans la
communauté, permettant de simuler des données de SNe~Ia et d'en calculer la
distance selon les hypothèses de corrélation~; nous y montrerons le biais
possible sur la mesure de $w$ selon les hypothèses de corrélation. Nous donnons
dans le dernier chapitre quelques perspectives à une possible suite de ces
travaux.

\vfill

Nous portons à l'attention de quiconque lit cette thèse que nous avons choisi
d'utiliser partiellement la grammaire neutre du français lorsque cela est jugé
nécessaire ou pertinent. Nous nous basons pour cela sur l'ouvrage et le travail
proposé par~\cite{alpheratz2018}, dont les résultats principaux sont visibles
sur son site~: \href{https://www.alpheratz.fr/linguistique/genre-neutre/}
{https://www.alpheratz.fr/linguistique/genre-neutre/}.

\vfill

% \bibliographystyle{../main/aa_url}
% \shorthandoff{:}
% \bibliography{../chapters/99_references}

\end{document}
