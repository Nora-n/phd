\documentclass[../main/main.tex]{subfiles}
\begin{document}

\chapter{Supernovae de type Ia}\label{ch:sne}

\epigraph{\openquote Il faut porter en soi un chaos pour pouvoir mettre
au monde une étoile dansante.\closequote}{\textsc{Nietzsche}, \textit{Ainsi
parlait Zarathoustra}}

%\dominitoc
%\tableofcontents
\vfill
\minitoc
\vfill
\newpage

\section{Fin de vie des étoiles}\label{sec:death}
\subsection{Classification}\label{ssec:class} % strolger2004, 2.2
\subsection{Physique de l'explosion}\label{ssec:explo}

\section{Propriétés}\label{sec:sneprop}
\subsection{Courbe de lumière}\label{ssec:lc}
\subsubsection*{$t_0$}
\subsubsection*{$c$}
\subsubsection*{$x_0$}
\subsubsection*{$x_1$}
\subsection{Spectroscopie}\label{ssec:spectro}

\section{Standardisation}\label{sec:stand}
\subsection{Corrélations}\label{ssec:corr}
\subsection{Modèle \texttt{SALT2.4}}\label{ssec:salt}

\section{Cosmologie avec les SNe~Ia}\label{sec:snecosmo}
\subsection{Diagramme de Hubble}\label{ssec:hubble}
\subsection{Détermination des paramètres cosmologiques}\label{ssec:pcosmo}
\subsection{Biais actuels}\label{ssec:biais}

Pour ces raisons, on étudie l'évolution des SNe~Ia

\newpage

% \dominilof
% \listoffigures
% \dominilot
% \listoftables
\minilof
\minilot

% \bibliographystyle{../main/aa_url}
% \shorthandoff{:}
% \bibliography{../chapters/99_references}

\end{document}
