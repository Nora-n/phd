\documentclass[../main/main.tex]{subfiles}
\begin{document}
\raggedbottom

\chapter{\'Evolution avec le redshift}\label{ch:stretch}

\epigraph{\openquote Citation test\closequote}{Autaire}

Comme présenté Chapitre~\ref{ch:sne}, la nature détaillée des SNe~Ia reste
incertaine, et à mesure que les statistiques des relevés augmentent, la question
des incertitudes systématiques astrophysiques se pose, notamment celle de
l'évolution des populations de SNe~Ia. De nombreuses études des propriétés
globales de galaxies hôtes ont montré des correlations significatives entre les
propriété d'un hôte et ses SNe~Ia \citep{kelly2010, hayden2013, roman2018},
amenant à l'utilisation répandue d'une marche de magnitude en fonction de la
masse de la galaxie dans la standardisation des SNe~Ia \citep{scolnic2018,
smith2020, popovic2021a}, les SNe dans une galaxie massive étant plus lumineuses
en moyenne. Dans leurs travaux, \cite{rigault2020} ont montré une
corrélation avec un traceur local

À cet effet, nous étudions la dépendance au redshift de l'étirement de courbe de
lumière issu d'un ajustement par \texttt{SALT2.4} de SNe~Ia, qui est une
propriété purement intrinsèque des SNe, afin de sonder sa dérive potentielle
avec le redshift.

% Il a été démontré que l'étirement des SN est fortement corrélé avec
% l'environnement des SN, notamment avec les traceurs de l'âge stellaire. Nous
% avons modélisé la distribution sous-jacente de l'étirement en fonction du
% décalage vers le rouge, en utilisant l'évolution de la fraction de SNe Ia jeunes
% et vieux telle que prédite par l'ensemble de données SNfactory, et en supposant
% une distribution sous-jacente de l'étirement constante pour chaque population
% d'âge constituée de mélanges gaussiens. Nous avons testé notre prédiction par
% rapport à des échantillons publiés qui ont été coupés pour avoir des effets de
% sélection de magnitude marginale, de sorte que tout changement observé est bien
% d'origine astrophysique et non observationnelle. Dans cette première étude, il y
% a des indications que la distribution d'étirement sous-jacente des SN Ia évolue
% en fonction du décalage vers le rouge, et que le modèle de dérive de l'âge est
% une meilleure description des données que tout modèle constant dans le temps, y
% compris les distributions asymétriques basées sur l'échantillon qui sont souvent
% utilisées pour corriger le biais de Malmquist à une signification supérieure à
% 5$\sigma$. Le modèle d'étirement sous-jacent favorisé est bimodal, composé d'un mode
% d'étirement élevé partagé par les environnements jeunes et anciens, et d'un mode
% d'étirement faible qui est exclusif aux environnements anciens. L'effet précis
% de l'évolution au redshift des propriétés intrinsèques d'une population de SN Ia
% sur la cosmologie reste à étudier. Cependant, la dérive astrophysique de la
% distribution d'étirement des SN affecte les corrections actuelles du biais de
% Malmquist et, par conséquent, les distances qui sont dérivées sur la base des SN
% qui sont affectés par les effets de sélection observationnels. Nous soulignons
% que ce biais augmentera avec les relevés couvrant des plages de redshift de plus
% en plus larges, ce qui est particulièrement important pour le Large Synoptic
% Survey Telescope.

\vfill

% \dominitoc
% \tableofcontents
\minitoc

\vfill

\newpage

\section{Concept d'\^age}\label{sec:age}

\subsection{Travaux précédents}\label{ssec:rates}

De nombreuses études discutent des distributions de taux de formation de SNe~Ia
et de délais entre la création de l'étoile (progéniteur) et son explosion en
supernova \citep[par exemple,][]{mannucci2005, scannapieco2005, sullivan2006,
smith2012, childress2014, maoz2014}. En effet, les deux scénarios décrits
Section~\ref{ssec:explo} menant à l'explosion d'une naine blanche en supernova
Ia se déroulent sur des échelles pouvant extrêmement varier. Ces scénarios
peuvent se produire sur toute une variété d'échelles de temps, de quelques
centaines de \si{Mans} à l'âge de l'Univers. Ceci s'observe par l'existence de
taux de formation de SNe~Ia différents en fonction de l'âge des progéniteurs, ce
qu'on appelle distribution du temps de retard \citep[DTD en anglais pour Delay
Time Distribution, voir][]{mannucci2006}~: certaines explosent rapidement après
un bref pic de formation stellaire.

\cite{maoz2012} indiquent obtenir le taux de formation de SNe~Ia «~simplement~»
en divisant le nombre total de supernovae acquises par un sondage par le temps
effectif pendant lequel chaque SN aurait pu être détectée (différent selon la
stratégie d'observation et l'efficacité du sondage).

% Mannucci 2005 alone
% In figure 3 we show that also these measurements are well reproduced by
% the two populations model introduced above. Figure 3 also shows that the slow
% exploders dominate the total rate at z< 1, while the fast events are more common
% before this epoch.

\subsection{Implications}\label{ssec:ageimpl}

Cette distinction entre SNe~Ia rapides («~jeunes~») et lentes («~vieilles~») en
fonction du taux de formation stellaire implique une évolution des
sous-populations en fonction du redshift, comme présenté
Section~\ref{sssec:snflssfr}. Nous pouvons notamment tracer l'évolution de la
fraction de jeunes étoiles en fonction du redshift, $\delta(z)$, donnée
Équation~\ref{eq:deltaz}, voir Figure~\ref{fig:deltaz}.

\begin{figure}[ht]
    \centering
    \includegraphics[width=\linewidth]{deltaz_hist.pdf}
    % \captionsetup{justification=centering}
    \caption[Évolution de la fraction de jeunes étoiles en fonction du redshift,
    $\delta(z)$.]{Évolution de la fraction de jeunes étoiles en fonction du
        redshift, $\delta(z)$. Le point noir indique la valeur fixée
        par~\cite{rigault2020}. En couleur sont les histogrammes complets des
    sondages SDSS, PS1 et SNLS de~\cite{scolnic2018}. Nous pouvons y lire que
les SNe~Ia de SDSS sont en moyenne dans un environnement composé
d'approximativement 60\% de jeunes supernovae, alors que celles à l'extrêmité
haute de l'intervalle de redshift couvert par SNLS sont dans un environnement
composé d'approximativement 80\% de jeunes étoiles.}
    \label{fig:deltaz}
\end{figure}

Mise à part cette évolution venant naturellement avec la définition du critère
jeune/vieux, ces deux sous-populations présentent des caractéristiques qui leur
sont propres. \cite{rigault2020} présentent dans leur travaux une différence de
magnitude basée sur cette dichotomie, trouvant que les vieilles SNe~Ia sont en
moyenne \SI{0.065}{mag} plus lumineuses que les jeunes. D'autres études
définissent des marches de magnitude selon d'autres critères, par exemple la
masse de l'hôte (voir Section~\ref{ssec:host}), mais la particularité de celle
basée sur l'âge est double~:
\begin{enumerate}
    \item Elle permet de définir une marche évoluant avec le redshift~;
    \item Elle permet de retrouver les valeurs déterminées par le biais d'autres
        traceurs comme démontré dans~\cite{briday2022}.
\end{enumerate}

Dans le cadre de cette thèse, c'est la distribution de l'étirement qui nous
intéresse. En effet, si ces deux sous-populations de SNe~Ia présentent des
distributions différentes, on s'attend à avoir une évolution de cette propriété
en moyenne, améliorant notre compréhension de ces objets. Bien que cela
n'intervienne pas directement dans l'amélioration du calcul des paramètres
cosmologiques qui utilisent la standardisation \textit{via} la relation de
\textsc{Tripp} (voir Section~\ref{sec:stand}), celle-ci se basant uniquement sur
la valeur de l'étirement sans supposer une quelconque distribution, cette
évolution impactera les études qui utilisent une distribution sous-jacente
d'étirement, comme c'est le cas dans le cadre de simulations de données où des
SNe~Ia sont générées à partir d'expressions mathématiques.

\section{Modélisation de l'étirement}\label{sec:xmod}

Pour modéliser l'évolution de la distribution complète de l'étirement des SNe en
fonction du redshift, nous devons modéliser la distribution de l'étirement des
SNe pour chaque sous-échantillon d'âge étant donné notre modèle susmentionné de
l'évolution de la fraction des SNe~Ia jeunes et vieilles avec le temps cosmique.
\cite{rigault2020} ont présenté la relation entre l'étirement des SN et la
mesure du LsSFR, un traceur de l'âge des progéniteurs, en utilisant
l'échantillon SNfactory. Cette relation est illustrée dans la
Fig.~\ref{fig:stretchlssfr} pour les SNe de SNfactory utilisés dans l'analyse
actuelle. Étant donné la structure du nuage de points étirement-LsSFR, notre
modèle de la distribution sous-jacente de l'étirement des SN~Ia est défini comme
suit~:
\begin{itemize}
    \item la distribution de l'étirement de la population la plus jeune
        ($\log(\mathrm{LsSFR})\geq-10,82$) est modélisée comme une distribution
        normale unique $\mathcal{N}(\mu_1, \sigma_1{}^2)$~;
    \item et la distribution de l'étirement de la population la plus âgée
        ($\log(\mathrm{LsSFR})<-10,82$) est modélisée comme un mélange gaussien
        bimodal $a\times \mathcal{N} (\mu_1, \sigma_1{}^2) + (1-a)\times
        \mathcal{N}(\mu_2, \sigma_2{}^2)$, où un mode est le même que pour la
        population jeune, $a$ représentant l'effet relatif des deux modes.
\end{itemize}

\begin{figure}
    \centering
    \includegraphics[width=\linewidth]{model_base_hist.pdf}
    \caption[Étirement en fonction du LsSFR des SNe~Ia de SNfactory et modèles
    d'étirement de référence ajustés]{\textit{Principal}: étirement de courbe de
        lumière ($x_1$) issu d'un ajustement par \textsc{\texttt{SALT2.4}} en
        fonction du LsSFR pour les SNe de SNfactory. La couleur correspond à la
        probabilité $p_y$ que la SN~Ia soit jeune, c'est-à-dire qu'elle ait
        $\log\mathrm{LsSFR} \geq -10.82$ \citep[voir][]{rigault2020}. \textit{À
        droite}: histogramme pondéré par $p_y$ des étirements des SNe, ainsi que
        le modèle de référence ajusté~; les contributions de la population jeune
    et âgée sont indiquées en violet et en jaune, respectivement.}
    \label{fig:stretchlssfr}
\end{figure}

La fonction de distribution de probabilité (pdf) de l'étirement d'une SN donnée
sera alors la combinaison linéaire des distributions d'étirement de ces deux
populations, pondérées par sa probabilité $y^i$ d'être jeune. De manière
générale cependant, la fraction de jeunes SNe~Ia est donnée par $\delta(z)$
(voir Équation~\ref{eq:deltaz}), et donc notre modèle de dérive avec le redshift
de la moyenne de la distribution sous-jacente d'étirement $X_1(z)$ est donnée
par~:
\begin{align}\label{eq:stretchz}
    X_1(z) = \delta(z)&\times \mathcal{N}(\mu_1,\sigma_1{}^2) + \nonumber \\
    (1-\delta(z))&\times \left[ a\times\mathcal{N}(\mu_1,\sigma_1{}^2) +
    (1-a)\times\mathcal{N}(\mu_2,\sigma_2{}^2) \right]
\end{align}
Ceci constitue notre modèle de dérive de référence.

\subsection{Paramétrisations}\label{ssec:pmod}

Compte tenu de la probabilité $y^i$ qu'une SN donnée soit jeune et supposant
notre modèle de référence (voir Section~\ref{sec:xmod}), la probabilité de
mesurer un étirement \texttt{SALT2.4} $x_1^i$ avec une erreur $\d x_1^i$ est
donné par~:
\begin{align}\label{eq:likelihoodsnf}
    \prob{x^i_1}{\overrightarrow{\theta}; \mathrm{d}x^i_1, y^i} =
    y^i & \times
    \mathcal{N}\left(x^i_1 \mid \mu_1, \sigma_1{}^2+\mathrm{d}x^i_1{}^2\right) +
    \nonumber\\
    (1-y^i) &\times \bigg[
    a \times \mathcal{N}\left(x^i_1 \mid \mu_1,
    \sigma_1{}^2+\mathrm{d}x^i_1{}^2\right) +
    \nonumber\\
    & (1-a) \times \mathcal{N}\left(x^i_1 \mid \mu_2,
    \sigma_2{}^{2}+\mathrm{d}x^i_1{}^2\right) \bigg]
\end{align}

L'estimation du maximum de vraisemblance des cinq paramètres libres
$\overrightarrow{\theta}\equiv({\mu_1, \mu_2, \sigma_1, \sigma_2,a})$ du modèle
s'obtient en ajustant l'équation suivante~:

\begin{equation}\label{eq:likelihood}
    -2\ln(L) = -2 \sum_i \ln \prob{x_1^i}{\overrightarrow{\theta};
    \mathrm{d}x_1^i, y^i}
\end{equation}
Selon que nous pouvons estimer $y^i$ directement à partir des mesures de LsSFR
ou non, il y a deux façons de procéder. Nous les décrivons ci-dessous.

\subsubsection*{Avec LsSFR}\label{sssec:lssfr}

Pour l'échantillon SNfactory, nous pouvons facilement fixer $y^i = p_y^i$, la
probabilité d'avoir $\log(\mathrm{LsSFR}) \geq -10,82$ (voir
Figure~\ref{fig:stretchlssfr}) afin de minimiser l'Équation~\ref{eq:likelihood}
par rapport à $\overrightarrow{\theta}$. Les résultats de l'ajustement de ce
modèle avec les SNe~Ia de SNf sont présentés dans la
Tableau~\ref{tab:modelresults} et illustrés Figure~\ref{fig:modelall}.

\begin{table*}
    \centering
    \caption[Valeurs des paramètres du modèle d'étirement de référence selon
    l'échantillon.]{Valeurs des paramètres issus des meilleurs ajustements du
        modèle de distribution de l'étirement de référence lorsqu'il est
        appliqué à l'ensemble de données de SNfactory seulement (114 SNe~Ia), à
        l'échantillon fiduciel (569 SNe~Ia) ou à l'échantillon conservatif
    (422).}
    \label{tab:modelresults}
    \begin{tabular}{lccccc}
        \toprule
        Échantillon & $\mu_1$             & $\sigma_1$
                    & $\mu_2$             & $\sigma_2$
                    & $a$ \\
        \midrule
        SNfactory   & $ 0.41 \pm 0.05$    & $0.55 \pm 0.04$
                    & $-1.38 \pm 0.07$    & $0.44 \pm 0.06$
                    & $ 0.48 \pm 0.06$ \\
        Fiduciel    & $ 0.37 \pm 0.04$    & $0.61 \pm 0.03$
                    & $-1.22 \pm 0.11$    & $0.56 \pm 0.07$
                    & $ 0.51 \pm 0.07$ \\
        Conservatif & $ 0.38 \pm 0.04$    & $0.60 \pm 0.03$
                    & $-1.26 \pm 0.09$    & $0.53 \pm 0.06$
                    & $ 0.47 \pm 0.06$ \\
        \bottomrule
    \end{tabular}
\end{table*}

\subsubsection*{Sans LsSFR}\label{sssec:z}

Lorsque les mesures directes de LsSFR font défaut (c'est-à-dire en absence de
$p_y^i$), nous pouvons étendre cette analyse aux autres échantillons que
SNfactory en utilisant l'évolution avec le redshift de la fraction $\delta(z)$
des jeunes SNe~Ia (Équation~\ref{eq:deltaz}) comme un indicateur alternatif de
la probabilité qu'une SN soit jeune. Cela implique toujours la minimisation de
l'Équation~\ref{eq:likelihood} par rapport aux paramètres 
$\overrightarrow{\theta}\equiv(\mu_1, \mu_2, \sigma_1, \sigma_2, a)$ de la
distribution d'étirement $X_1$ (Équation~\ref{eq:likelihoodsnf}) mais en
supposant cette fois que $y^i = \delta(z^i)$ pour une SN $i$ donnée.

Pour le reste de cette analyse, nous avons ainsi ajusté
l'Équation~\ref{eq:likelihood} en utilisant $p_y^i$ la probabilité que la SN $i$
soit jeune lorsqu'elle est disponible (c'est-à-dire pour les données de
SNfactory) et $\delta(z^i)$, la fraction attendue de jeunes SNe~Ia au redshift
$z^i$ de la SN sinon.\bigbreak

Les résultats de l'ajustement de ce modèle à l'ensemble des 569 (respectivement
422) SNe~Ia de l'échantillon fiduciel (conservatif) sont présentés dans la
Tableau~\ref{tab:modelresults}, et l'évolution prédite de l'étirement avec le
redshift ($x_1$ attendu compte tenu de la distribution de
l'équation~\ref{eq:stretchz}) est illustrée sous la forme d'une bande bleue dans
la Figure~\ref{fig:modelall} qui tient compte des erreurs des paramètres et de
leurs covariances. Cette figure montre que l'étirement moyen mesuré des SNe~Ia
par intervalle de redshift (contenant tous le même nombre de données) suit de
près notre modélisation de la dérive avec le redshift. C'est en effet ce que
l'on attend si les environnements vieux favorisent les faibles étirements de SN
\citep[voir par exemple][]{howell2007} et si la fraction de vieilles SNe~Ia
diminue en fonction du redshift. Nous discutons quantitativement de ces
résultats Section~\ref{sec:xres}.

\begin{figure}
    \centering
    \includegraphics[width=\linewidth]{stretchevol_all_vs_snf-mean.pdf}
    \caption[Évolution de l'étirement moyen des SNe~Ia en fonction du redshift
    issu de la prédiction de notre modèle de référence selon l'échantillon
    utilisé.]{Évolution de l'étirement moyen ($x_1$) des SNe~Ia issus d'un
        ajustement \texttt{SALT2.4} en fonction du redshift pour notre modèle de
        référence. Les marqueurs montrent la simple moyenne de l'étirement (les
        erreurs sont estimées à partir de la dispersion) mesurée dans des
        intervalles de redshift de tailles d'échantillon égales, indiquées en
        gris clair en bas de chaque intervalle. Les marqueurs opaques et
        transparents sont utilisés lorsque les échantillons fiduciel ou
        conservatif sont considérés, respectivement. La ligne horizontale orange
        représente l'étirement moyen pour le modèle Gaussien sans dérive
        (dernière ligne de le Tableau~\ref{tab:comp}) sur l'échantillon fiduciel.
        Les meilleurs ajustements de notre modèle de dérive de référence sont
        présentés en bleu, bleu pointillé et gris lorsqu'ils sont ajustés sur
        l'échantillon fiduciel, conservatif ou l'ensemble de données SNfactory
        uniquement, respectivement~; ils sont tous compatibles entre eux et avec
        les données. La bande bleu clair illustre l'amplitude de l'erreur
        (covariance comprise) du modèle le mieux ajusté lorsqu'on considère
    l'ensemble de données fiduciel.}
    \label{fig:modelall}
\end{figure}

\subsection{Implémentation}\label{ssec:modimpl}

Dans la Section~\ref{ssec:pmod}, nous avons modélisé la distribution
sous-jacente de l'étirement des SNe~Ia en suivant~\cite{rigault2020},
c'est-à-dire avec une unique Gaussienne pour les jeunes SNe et un mélange de
deux Gaussiennes pour la population des vieilles SNe~Ia, la première étant la
même que pour la jeune population et la seconde une qui est spécifique aux SNe à
déclin rapide qui semblent n'exister que dans les environnements localement
vieux. C'est ce que nous appelons notre modèle de référence. Cependant, pour
tester différents choix de modélisation, nous avons mis en œuvre une suite de
paramétrisations alternatives que nous avons également ajustées aux données en
suivant la procédure décrite dans la Section~\ref{ssec:pmod}.

\cite{howell2007} ont utilisé un modèle unimodal plus simple par catégorie
d'âge, en supposant une distribution normale unique pour chacune des populations
jeune et âgée. Nous avons donc considéré un modèle «~Howell+dérive~», comportant
une seule Gaussienne par groupe d'âge et intégrant la dérive avec $\delta(z)$ de
l'équation~\ref{eq:deltaz}.

Alternativement, comme nous cherchons à vérifier l'existence d'une évolution
avec le redshift, nous avons également testé des modèles constants en limitant
les modèles de référence et de \textsc{Howell} à utiliser une fraction de jeunes
SNe~Ia $\delta(z) \equiv f$ indépendante du redshift~; ces modèles sont appelés
ci-après «~référence+constant~» et «~Howell+constant~».

Nous avons également considéré un autre modèle intrinsèquement non-dérivant, la
forme fonctionnelle développée pour la méthode \textit{BEAMS with Bias
Correction}~\citep[BBC,][]{scolnic2016, kessler2017}, utilisée dans les analyses
cosmologiques utilisant les SNe~Ia les plus récentes~\citep[par
exemple][]{scolnic2018, abbott2019, riess2016, riess2019} pour tenir compte des
biais de \textsc{Malmquist}. Le formalisme de BBC suppose des distributions
d'étirement Gaussiennes asymétriques basées sur la forme de chaque échantillon
(et donc intrinsèquement sans dérive)~: $\mathcal{N}\left(\mu, \sigma_-{}^2\;
\text{si} \;x_1<\mu,\; \text{sinon} \;\sigma_+{}^2\right)$. L'idée derrière
cette approche par échantillon est double~\citep{scolnic2016, scolnic2018}~:
\begin{enumerate}
    \item Les biais de \textsc{Malmquist} sont déterminés par les propriétés des
        relevés~;
    \item Comme les relevés actuels couvrent des plages de redshift limitées,
        une approche par échantillon couvre certaines informations potentielles
        sur l'évolution avec le redshift.
\end{enumerate}
Une discussion plus détaillée sur BBC se trouve Section~\ref{ssec:disc}. Enfin,
par souci d'exhaustivité, nous avons également considéré des modèles Gaussiens
purs en asymétriques indépendants du redshift.

\section{Résultats}\label{sec:xres}

Nous exposons dans cette section les résultats quantitatifs de cette étude
Section~\ref{ssec:xcomp} et proposons une discussion de ceux-ci
Section~\ref{ssec:disc}.

\subsection{Comparaison aux données}\label{ssec:xcomp}
Nous avons ajusté chacun des modèles décrits ci-dessus sur les échantillons
fiduciel et conservatif (voir Chapitre~\ref{ch:sample}). Les résultats sont
rassemblés dans le Tableau~\ref{tab:comp} et sont illustrés
Figure~\ref{fig:mod_comp}.

\begin{table}[ht]
    \centerfloat
    \begin{threeparttable}
        \caption[Comparaison de la capacité relative de chaque modèle à décrire
        les données par rapport au modèle de référence]{Comparaison de la
            capacité relative de chaque modèle à décrire les données par rapport
        au modèle de référence.}\label{tab:comp}
        \begin{tabular}{lcccccccc}
            \toprule
            & & & \multicolumn{3}{c}{Échantillon fiduciel (569 SNe)}
                & \multicolumn{3}{c}{Échantillon conservatif (422 SNe)} \\
            \cmidrule(lr){4-6} \cmidrule(lr){7-9}
            Nom & dérive & $k$ &
            $-2\ln(L)$ & AIC & $\Delta$AIC & $-2\ln(L)$ & AIC & $\Delta$AIC\\
            \midrule
            Référence & $\delta(z)$ & 5
            & 1456,7 & 1466,7 & -- 
            & 1079,5 & 1089,5 & -- \\
            Howell+dérive & $\delta(z)$ & 4
            & 1463,3 & 1471,3 & $-4,6$
            & 1088,2 & 1096,2 & $-6,7$ 
            \\
            Asymétrique & -- & 3
            & 1485,2 & 1491,2 & $-24,5$
            & 1101,3 & 1107,3 & $-17,8$ 
            \\
            Howell+constant & $f$ & 5
            & 1484,2 & 1494,2 & $-27,5$
            & 1101,2 & 1111,2 & $-21,7$ 
            \\
            Référence+const & $f$ & 6
            & 1484,2 & 1496,2 & $-29,5$
            & 1101,2 & 1113,2 & $-23,7$ 
            \\
            Asym.\ par échant. & Par échant. & 3$\times$5
            & 1468,2 & 1498,2  & $-31,5$
            & 1083,6 & 1113,6  & $-24,1$ 
            \\
            Gaussienne & -- & 2
            & 1521,8 & 1525,8 & $-59,1$
            & 1142,6 & 1146,6 & $-57,1$ 
            \\
            \bottomrule
        \end{tabular}
        \begin{tablenotes}[flushleft]
            \item \textbf{\hspace{-3.2pt}Notes.} Pour chaque modèle considéré,
                nous indiquons si le modèle dérive ou non, son nombre de
                paramètres libres $k$, et pour les échantillons fiduciel et
                conservatif, $-2\ln(L)$ (voir Équation~\ref{eq:likelihood}),
                l'AIC et la différence d'AIC ($\Delta$AIC) entre ce modèle et le
                modèle de référence, choisi car présentant l'AIC le plus faible.
        \end{tablenotes}
    \end{threeparttable}
\end{table}

\begin{figure}[ht]
    \centering
    \includegraphics[width=\linewidth]{mod_comp.pdf}
    \caption[$\Delta$AIC entre le modèle de base et les autres
    modèles]{$\Delta$AIC entre le modèle de référence et les autres
        modèles (voir Tableau~\ref{tab:comp}). Les marqueurs bleus pleins et
        ouverts correspondent aux modèles avec et sans dérive du redshift,
        respectivement. Les marqueurs transparents montrent les résultats
        lorsque l'analyse est effectuée sur l'échantillon conservatif plutôt que
        sur l'échantillon fiduciel. Les bandes de couleur illustrent la validité
        des modèles, d'acceptable ($\Delta$AIC > -5) à exclu ($\Delta$AIC <
        -20), voir le corps de texte. En suivant ces valeurs d'AIC, tous les
        modèles sans dérive (marqueurs ouverts) sont exclus car il représentent
    moins bien les données que le modèle de référence, avec dérive.}
    \label{fig:mod_comp}
\end{figure}

Parce que les divers modèles présentent différents degrés de liberté, nous avons
utilise le critère d'information d'\textsc{Akaike}~\citep[AIC, voir par
exemple][]{burnham2004} pour comparer leur capacité à décrire correctement les
observations. Cet estimateur pénalise l'ajout de degrés de liberté
supplémentaire afin d'éviter un ajustement excessif des données. Il est défini
comme suit~:
\begin{equation}\label{eq:aic}
    \mathrm{AIC} = -2\ln(L) + 2k,
\end{equation}
où $-2\ln(L)$ est obtenu en minimisant l'Équation~\ref{eq:likelihood}, et $k$
est le nombre de paramètres libres à ajuster. Le modèle de référence est celui
de plus petit AIC~; par rapport à ce modèle, les modèles avec $\Delta$AIC > -5
sont qualifiés d'acceptables, ceux avec $-5 > \Delta\mathrm{AIC} > -20$ ne sont
pas favorisés et ceux avec $\Delta$AIC < -20 sont jugés exclus. Cela correspond
approximativement aux limites de 2, 3 et 5$\sigma$ pour une distribution de
probabilité Gaussienne.

Le meilleur modèle (avec le plus petit AIC) est le modèle dit de base et
constitue donc notre modèle de référence~; ceci est vrai pour les échantillons
fiduciel et conservatif. Le modèle de base a également le plus petit $-2\ln(L)$,
ce qui en fait le modèle le plus probable même si l'on ne tient pas compte de la
question de l'ajustement excessif qui est pris en compte par le formalisme de
l'AIC.

En outre, nous constatons que les distributions d'étirement indépendantes du
redshift sont toutes exclues comme descriptions appropriées des données
relativement au modèle de base. Le meilleur modèle  non-dérivant (le modèle
asymétrique) a une chance très marginale ($p \equiv \exp(\Delta\mathrm{AIC}/2) =
\num{5e-6}$) de décrire les données aussi bien que le modèle de base. Ce
résultat n'est qu'une évaluation quantitative de faits qualitatifs qui sont
clairement visibles sur la Figure~\ref{fig:modelall}~: l'étirement moyen des SNe
par intervalle de redshift suggère fortement une évolution significative du
redshift plutôt qu'une valeur constante, et cette évolution est bien décrite par
l'Équation~\ref{eq:deltaz}.

De manière surprenante, la modélisation asymétrique Gaussienne par échantillon
utilisée par les implémentations actuelles de la technique BBC
\citep{scolnic2016, kessler2017} présente l'une des valeurs d'AIC les plus
élevées de notre analyse. Bien que son $-2\ln(L)$ soit le plus petit de tous les
modèles indépendants du redshift (mais toujours inférieur de \num{11.5} au
modèle de référence), il est fortement pénalisé car il nécessite 15 paramètres
libres ($\mu_0$, $\sigma_{\pm}$ pour chacun des cinq échantillons de l'analyse).
Par conséquent, il en résulte un $\Delta\mathrm{AIC} < -20$, ce qui pourrait
être interprété comme une probabilité $p = \num{2e-7}$ d'être une aussi bonne
représentation des données que le modèle de référence.

Nous notons que lorsque l'on compare des modèles qui ont été ajustés sur des
sous-échantillons individuels plutôt que globalement, le critère d'information
bayésien ($\mathrm{BIC} = -2\ln(L) + k\ln(n)$, avec $n$ le nombre de points de
données) pourrait être pls adapté que l'AIC car il tient explicitement compte du
fait que chaque sous-échantillon est ajusté séparément~: le BIC du modèle par
échantillon est alors la somme des BIC de chaque échantillon. Nous trouvons
$\Delta\mathrm{BIC} = -48$, ce qui réfute également le modèle Gaussien
asymétrique par échantillon comme étant aussi pertinent que notre modèle de
référence.

Afin de s'assurer que nos résultats ne sont pas influencés par le
sous-échantillon HST incomplètement modélisé, nous avons recalculé le
$\Delta$AIC pour chaque modèle en excluant cet ensemble de données~; cela n'a
pas modifié le $\Delta$AIC de plus que quelques dixièmes. La cohérence de ces
valeurs avec celles du Tableau~\ref{tab:comp} montre que le sous-échantillon HST
n'influence pas nos conclusions.

Nous rapportons dans le Tableau~\ref{tab:bbc} notre détermination de $\mu_0$ et
$\sigma_{\pm}$ pour chaque échantillon lorsqu'un modèle Gaussien asymétrique a
été appliqué, et ajusté sur les échantillons normalement sans effets de
sélection en utilisant nos coupes fiducielles (voir Chapitre~\ref{ch:sample}).
Nos résultats sont en accord étroit avec ceux de~\cite{scolnic2016} pour SNLS et
SDSS et avec les résultats rapportés par~\cite{scolnic2018} pour PS1, qui ont
dérivé ces paramètres de modèle en utilisant le formalisme complet BBC, qui
utilise de nombreuses simulations pour modéliser les effets de sélection
observationnels \citep[voir les détails par exemple Section~3 de][et le
Chapitre~\ref{ch:snana}]{kessler2017}. L'accord entre notre ajustement des
Gaussiennes asymétriques sur la partie supposée sans effets de sélection des
échantillons et les résultats dérivés en utilisant le formalisme BBC soutient
notre approche pour construire un échantillon avec des effets de sélection
observationnels négligeables. Si nous devions utiliser les valeurs les mieux
ajustées de~\cite{scolnic2016} et de~\cite{scolnic2018} pour les paramètres
asymétriques $\mu_0,\sigma_{\pm}$ pour les échantillons SNLS, SDSS et PS1
respectivement, le $\Delta$AIC entre notre modèle modèle de dérive de référence
et la modélisation BBC irait encore plus loin, passant de -32 à -47. Nous
discutons plus en détail de la conséquence de ce résultat pour la cosmologie
dans la Section~\ref{ssec:disc}.

\begin{table}
    \centering
    \begin{threeparttable}
        \caption[Paramètres de meilleur ajustement pour notre modélisation
        asymétrique par échantillon de la distribution d'étirement
        sous-jacente.]{Paramètres de meilleur ajustement pour notre modélisation
        asymétrique par échantillon de la distribution d'étirement sous-jacente.}
        \label{tab:bbc}
        \begin{tabular}{lcccccc}
        \toprule
        Asymétrique   & $\sigma_{-}$    & $\sigma_{+}$    & $\mu_0$ \\
        \midrule
        SNfactory     & 1,34 $\pm$ 0,13 & 0,41 $\pm$ 0,10 & 0,68 $\pm$ 0,15 \\
        SDSS\tnote{1} & 1,31 $\pm$ 0,11 & 0,42 $\pm$ 0,09 & 0,72 $\pm$ 0,13 \\
        PS1\tnote{2}  & 1,01 $\pm$ 0,11 & 0,52 $\pm$ 0,12 & 0,38 $\pm$ 0,16 \\
        SNLS\tnote{3} & 1,41 $\pm$ 0,13 & 0,15 $\pm$ 0,13 & 1,22 $\pm$ 0,15 \\
        HST           & 0,76 $\pm$ 0,36 & 0,79 $\pm$ 0,35 & 0,11 $\pm$ 0,44 \\
        \bottomrule
        \end{tabular}
        \begin{tablenotes}[flushleft]
            \item \textbf{\hspace{-3,2pt}Notes.} On compare ces valeurs à celles
                des études intégrant les mêmes distributions asymétriques.
            \item [1] $\sigma_{-} = 1,65 \pm 0,08$, $\sigma_{+} = 0,10 \pm
                0,10$, $\mu_0 = 1,14 \pm 0,03$ \citep[Tableau~1,][]{scolnic2016}
            \item [2] $\sigma_{-} = 0,96 \pm 0,16$, $\sigma_{+} = 0,51 \pm
                0,14$, $\mu_0 = 0,37 \pm 0,21$ \citep[Tableau~3,][]{scolnic2018}
            \item [3] $\sigma_{-} = 1,23 \pm 0,10$, $\sigma_{+} = 0,28 \pm
                0,10$, $\mu_0 = 0,96 \pm 0,14$ \citep[Tableau~1,][]{scolnic2016}
        \end{tablenotes}
    \end{threeparttable}
\end{table}

Nous avons également effectué des tests permettant au mode d'étirement élevé de
la population âgée de différer de celui de la population jeune, ajoutant ainsi
deux degrés de liberté ($\mu_3, \sigma_3$). L'ajustement correspondant n'est pas
significativement meilleur, avec un $\Delta$AIC de \num{-0.4}. Cela renforce
notre hypothèse selon laquelle les populations jeune et âgée semblent
effectivement partager le même mode sous-jacent d'étirement élevé. De plus, nous
pouvons nous demander si un mode de faible étirement pourrait également exister
dans la population jeune (voir la Figure~\ref{fig:stretchlssfr}). Nous avons
testé cette hypothèse en permettant à cette population d'être également
bimodale, et nous avons constaté que l'amplitude d'un tel mode de faible
étirement était compatible avec 0 (< 2\%) dans cette jeune population. Plus
généralement, cela soulève la question de savoir dans quelle mesure un traceur
environnemental (ici le LsSFR) trace l'âge. Cette question a reçu une analyse
dédiée dans~\cite{briday2022}.

Enfin, le fait d'ignorer les mesures du LsSFR, qui ne sont disponibles que pour
l'ensemble de données SNfactory (voir la Section~\ref{sec:pmod}), réduit la
pertinence des résultats présentés dans cette section, comme prévu. Malgré cela,
les modèles non-dérivants restent fortement défavorisés. Voir
Figure~\ref{fig:mod_comp_zonly} et le Tableau~\ref{tab:comp_zonly}. Par exemple,
le modèle asymétrique Gaussien par échantillon le mieux ajusté est toujours
$\Delta$AIC < -10, ce qui est moins représentatif des données que notre modèle
de dérive de référence. On note que les seuls modèles impactés par l'absence de
LsSFR dans leur ajustement sont les modèles dérivants~: fixer la fraction de
jeunes SNe~Ia rend les modèles insensible à ce paramètre. Ces résultats
permettent de marquer la forte information apportée par ce traceur, expliquant
pourquoi les études de distributions sous-jacentes n'utilisant pas le LsSFR sont
moins clivantes que celle-ci.

\sidecaptionvpos{figure}{t}
\begin{SCfigure}[0.8][ht]
    \centering
    \includegraphics[width=.6\linewidth]{mod_comp_zonly.pdf}
    \caption[$\Delta$AIC entre le modèle de base et les autres modèles sans
    utiliser le LsSFR]{$\Delta$AIC entre le modèle de référence et les autres
        modèles sans utiliser le LsSFR (voir Tableau~\ref{tab:comp_zonly}). La
        légende est la même qu'en Figure~\ref{fig:mod_comp}. En revanche, la
        robustesse des résultats concernant l'inaptitude des modèles
        non-dérivants à représenter correctement les données diminue, même si
        les meilleurs modèles sont toujours ceux incluant une dérive (marqueurs
    pleins).}\label{fig:mod_comp_zonly}
\end{SCfigure}
\sidecaptionvpos{figure}{c}

\begin{table}[ht]
    \centerfloat
    \begin{threeparttable}
        \caption[Comparaison de la capacité relative de chaque modèle à décrire
        les données par rapport au modèle de référence sans utiliser le
        LsSFR]{Comparaison de la capacité relative de chaque modèle à décrire
            les données par rapport au modèle de référence sans utiliser le
        LsSFR.}\label{tab:comp_zonly}
        \begin{tabular}{lccccc}
            \toprule
            & & & \multicolumn{3}{c}{Échantillon fiduciel (569 SNe)} \\
            \cmidrule(lr){4-6}
            Nom & dérive & $k$ &
            $-2\ln(L)$ & AIC & $\Delta$AIC\\
            \midrule
            Référence & $\delta(z)$ & 5
            & 1477,8 & 1487,8 & --
            \\
            Howell+dérive & $\delta(z)$ & 4
            & 1479,3 & 1487,3 & $+0,5$
            \\
            Asymétrique & -- & 3
            & 1485,2 & 1491,2 & $-3,4$
            \\
            Howell+constant & $f$ & 5
            & 1484,2 & 1494,2 & $-6,4$
            \\
            Référence+const & $f$ & 6
            & 1484,2 & 1496,2 & $-8,4$
            \\
            Asym.\ par échant. & Par échant. & 3$\times$5
            & 1468,2 & 1498,2  & $-10,9$
            \\
            Gaussienne & -- & 2
            & 1521,8 & 1525,8 & $-38,0$
            \\
            \bottomrule
        \end{tabular}
        \begin{tablenotes}[flushleft]
            \item \textbf{\hspace{-3.2pt}Notes.} Dans cette étude sans LsSFR, on
                remarque que seuls les deux modèles dérivants changent de valeur
                par rapport au Tableau~\ref{tab:comp}~: fixer la fraction de
                jeunes étoiles rend les modèles insensibles à l'utilisation du
                LsSFR. Dans cette disposition, le modèle de référence arrive
                second au niveau de l'AIC, mais toujours premier au classement
                par $-2\ln(L)$.
        \end{tablenotes}
    \end{threeparttable}
\end{table}

\subsection{Discussion}\label{ssec:disc}

À notre connaissance, la modélisation de la dérive des redshifts des SNe~Ia n'a
jamais été explicitement utilisée dans les analyses cosmologiques, bien qu'un
formalisme de hiérarchie bayésienne tel que UNITY \citep{rubin2015}, BAHAMAS
\citep{shariff2016} ou Steve \citep{hinton2019} puisse facilement le permettre
\citep[voir, par exemple, les sections 1.3 et 2.5 de][]{rubin2015}. Ne pas le
faire constitue un problème de second ordre pour la cosmologie avec les SNe~Ia
car cela n'affecte que la manière dont le biais de \textsc{Malmquist} est pris
en compte. Tant que le paramètre de normalisation $\alpha$ de la relation de
\textsc{Phillips} \citep{phillips1993} ne dépend pas du redshift \citep[une
étude qui dépasse le cadre de cette thèse, mais voir, par
exemple][]{scolnic2018}, les magnitudes corrigées de l'étirement utilisées pour
la cosmologie sont effectivement insensibles à la distribution d'étirement
sous-jacente pour les échantillons complets. Cependant, les enquêtes présentent
généralement un biais de \textsc{Malmquist} significatif pour la moitié
supérieure de leur distribution de redshift de SNe. Par conséquent, une mauvaise
modélisation de la distribution d'étirement sous-jacente biaisera les magnitudes
dérivées des SNe de ces études.

Les techniques de correction du biais de \textsc{Malmquist} couramment
utilisées, telles que le formalisme BBC, supposent des fonctions Gaussiennes
asymétriques par échantillon pour modéliser les distributions d'étirement et de
couleur sous-jacentes. Comme le montre la Section~\ref{sec:xres} cependant, une
telle distribution par échantillon est exclue par rapport à notre modèle de
dérive. Contrairement à ce que \citep[Section 2]{scolnic2016} et \citep[Section
5.4]{scolnic2018} ont suggéré, à savoir que les études traditionnelles couvrent
des plages de redshift suffisamment limitées pour que l'approche par échantillon
tienne compte des dérives implicites du redshift, une modélisation directe de la
dérive avec le redshift est donc plus appropriée qu'une approche par
échantillon. Nous ajoutons ici qu'au fur et à mesure que les relevés
cosmologiques modernes tentent de couvrir des plages de redshift de plus en plus
larges afin de réduire les incertitudes systématiques de calibration, cette
approche par échantillon devient moins valide, notamment pour PS1, le
\textit{Dark Energy Survey} \citep[DES,][]{abbott2019}, et, bientôt, le LSST
\citep{ivezic2019}.

Nous illustrons Figure~\ref{fig:bbc_pdf_ps1} la différence de prédiction de la
distribution de l'étirement sous-jacent entre la modélisation asymétrique par
échantillon et notre modèle de dérive de référence pour l'échantillon PS1. Notre
modèle est bimodal, et l'amplitude relative de chaque mode dépend de la fraction
de jeunes et vieilles SNe~Ia dans l'échantillon en fonction du redshift~: plus
la fraction de vieilles SNe~Ia est élevée (à un faible redshift), plus
l'amplitude du mode d'étirement faible spécifique aux vieilles SNe~Ia est élevé.
Cette dépendance avec le redshift des distributions d'étirement sous-jacentes
est représentée par des couleurs allant du bleu au rouge sur la
Figure~\ref{fig:bbc_pdf_ps1} pour la gamme de redshift couvert par l'ensemble de
PS1. L'histogramme des $x_1$ observés suit le modèle que nous avons défini en
utilisant la somme des distributions sous-jacentes individuelles au redshift de
chaque SN du sondage (en noir). Comme prévu, les deux approches de modélisation
diffèrent surtout dans la partie négative de la distribution des étirements de
SNe. La distribution Gaussienne asymétrique passe par le milieu de la
distribution bimodale, surestimant le nombre de SNe~Ia à $x_1 \approx 0,7$ et
le sous-estimant à $x_1 \approx 1,7$ par rapport à notre modèle de dérive de
référence pour les redshifts typiques des SNe de PS1. Cela signifie que la
magnitude standardisée corrigée du biais d'une SN estimée à un redshift affecté
par la sélection observationnelle serait biaisée par une mauvaise modélisation
de la véritable distribution d'étirement sous-jacente.

\begin{figure}
    \centering
    \includegraphics[width=\linewidth]{bbc_comp_PS1_hist-nr.pdf}
    \caption[Comparaison des modélisations de BBC et de notre modèle de
    référence sur l'histogramme des étirements de PS1.]{Distribution de
        l'étirement des SNe~Ia de PS1 issus d'un ajustement \texttt{SALT2.4}
        ($x_1$) pour toutes les données du sondage, au-delà de notre limite
        fiducielle de redshift (histogramme gris). Cette distribution est
        supposée être un tirage aléatoire de la distribution d'étirement
        sous-jacente. Les lignes vertes montrent le modèle BBC de cette
        distribution sous-jacente (Gaussienne asymétrique). La ligne pleine (et
        sa bande) est notre meilleur ajustement (et son erreur)~; la ligne
        pointillée montre le résultat de~\cite{scolnic2018}. La ligne noire (et
        sa bande) montre notre modélisation de référence la mieux ajustée (et
        son erreur, voir Tableau~\ref{tab:modelresults}) qui inclut la dérive du
        redshift. À titre d'illustration, nous montrons (coloré du bleu au rouge
        avec des redshifts croissants) l'évolution de la distribution
        d'étirement sous-jacente en fonction du redshift pour la plage de
    redshift couverte par toutes les données de PS1.}
    \label{fig:bbc_pdf_ps1}
\end{figure}

L'évaluation de l'amplitude de ce biais de magnitude pour la cosmologie fait
l'objet du Chapitre~\ref{ch:snana}, en utilisant notre modèle de référence
(Équation~\ref{eq:stretchz}) à la place du modèle par échantillon. Cependant,
nous avons déjà mis en évidence que même si un modèle par échantillon sans
dérive pouvait donner des résultats comparables dans la partie limitée en volume
des différents échantillons, ces modèles seraient différents lorsqu'ils
seraient extrapolés à des redshifts plus élevés, précisément là où la
distribution sous-jacente sera important pour corriger les biais de
\textsc{Malmquist}.

À l'ère de la cosmologie moderne, où nous visons à mesurer $w_0$ à un niveau
inférieur au pourcentage et $w_a$ avec une précision de 10\% \citep[par
exemple,][]{ivezic2019}, nous soulignons que la modélisation correcte de la
dérive potentielle avec le redshift des SNe~Ia doit être étudiée plus
profondément et qu'il faut faire attention lorsque l'on utilise des échantillons
qui sont affectés par des effets de sélection observationnels.

\subsection{Prédictions et amélioration}\label{ssec:xpred}

Nous avons considéré une modélisation simple par mélange Gaussien à deux
populations. Des données supplémentaires exemptes de biais de \textsc{Malmquist}
significatifs nous permettraient de l'affiner. Notamment, les données aux
extrémités à bas et haut redshifts du diagramme de \textsc{Hubble} sont
particulièrement utiles pour l'analyse de cette dérive. Si les programmes de
relevé de SNe~Ia à haut redshifts Subaru et SeeChange ne sont pas encore
disponibles, les données fournies par la \textit{Zwicky Transient Facility}
\citep[ZTF,][]{bellm2019, graham2019} nous permettent d'établir un échantillon
extrêmement riche à bas-redshift (2000 SNe~Ia à $z \lesssim 0,7$).

\newpage

% \dominilof
% \listoffigures
% \dominilot
% \listoftables
\minilof
\minilot

% \bibliographystyle{../main/aa_url}
% \shorthandoff{:}
% \bibliography{../chapters/99_references}

\end{document}
