\documentclass[../main/main.tex]{subfiles}
\begin{document}

\chapter*{Résumé}\label{ch:resume}

Les Supernovae de Type Ia (SNe~Ia) sont des corps célestes de luminosité
transitoire résultant de l'explosion d'étoiles. Elles sont de nos jours au cœur
des analyses de cosmologie observationnelle par leur régularité dans leur
luminosité libérée, ce qui permet à différents sondages et télescopes d'en
mesurer la distance et donc le taux d'expansion de l'Univers.

Cependant, la nature détaillée des SNe Ia reste incertaine, et ces études
reposent sur des lois empiriques et notamment sur la distinction en deux
populations de SNe~Ia qui auraient des propriétés différentes. À mesure que les
statistiques des relevés augmentent, la question des incertitudes systématiques
astrophysiques se pose, notamment celle de l'évolution des populations de
SNe~Ia.

Dans cette perspective, nous implémentons des tentatives d'amélioration de notre
connaissance de la physique des SNe~Ia par le biais de l'étude de corrélations
entre leurs propriétés et leur environnement. Nous avons montré l'existence d'un
biais en lien avec la masse globale de la galaxie hôte d'une SN, et mis en
évidence l'exitence de sous-populations basées sur l'âge qui pourraient être
plus pertinentes en tant que traceur de la différence des propriétés observées
dans les SNe.

Notre thèse s'appuie sur cette hypothèse et le lien établi par des études
précédentes entre l'étirement des SNe et leur âge. Dans cette thèse, nous
étudions la dépendance au redshift de l'étirement de courbe de lumière issu d'un
ajustement par \texttt{SALT2} de SNe Ia, qui est une propriété purement
intrinsèque des SNe, afin de sonder sa dérive potentielle avec le redshift. Nous
modélisons différentes dépendances et donnons les résultats de notre analyse :
nous y verrons que la dérive astrophysique des propriétés des SNe~Ia est
fortement favorisée et que les modèles de distribution sous-jacente d'étirements
constants avec le redshift sont exclus comme étant de bonnes représentations des
données par rapport à notre modèle de référence.

L'impact de cette modélisation sur la détermination des paramètres cosmologiques
a été étudiée par le biais de simulations numériques, et indiquent un biais
jusqu'à 4\% de la valeur du paramètre d'état de l'énergie sombre, $w$, si ces
corrélations ne sont pas prises en compte.

\end{document}
