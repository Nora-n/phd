\documentclass[../main/main.tex]{subfiles}
\begin{document}

% \chapter*{Conclusions}\label{ch:conc}
% \addcontentsline{toc}{chapter}{Conclusions}

\appch{Conclusions}{ch:conc}

Au fil de cette thèse, nous avons montré comment procéder à l'étude précise et
approfondie d'un phénomène pouvant apporter un biais notable à nos mesures de
paramètres cosmologiques. En effet, la cosmologie observationnelle est par
essence une science non-reproductible et la nature imprécise des SNe~Ia force
les études à se reposer sur des lois empiriques dont les motivations sont
parfois approximatives. C'est à partir de la première standardisation de leur
luminosité avec leur étirement et leur couleur qu'elles ont pu être utilisées
comme des sondes utiles, et maintenant indispensables, à la compréhension de
l'Univers.

Afin de continuer à améliorer cette standardisation, la recherche avec les
SNe~Ia étudie les meilleurs paramètres permettant de les décrire. Certains
travaux augmentent le nombre de paramètres décrivant leurs variabilités
\citep[c'est le cas de SUGAR, décrit dans][]{leget2020}, s'écartant ainsi des
descriptions classiques de \texttt{SALT2}, d'autres implémentent des traceurs
environnementaux jugés plus pertinents à leur étude \citep[par exemple, le LsSFR
dans][ou la masse de la galaxie hôte dans~\citet{childress2013}]{rigault2013} qui
permettraient de distinguer deux populations au sein des SNe~Ia.
\cite{briday2022} se sont intéressæs à la qualité de ces différents traceurs
afin de déterminer lequel amène à leur meilleure caractérisation. Mais si les
possibilités d'amélioration de description et de compréhension sont nombreuses,
la dérive des propriétés mêmes des SNe~Ia avec le redshift reste peu étudiée.

À cet effet, notre thèse s'est focalisée à tester cette hypothèse en se basant
sur la fonction d'évolution de l'âge moyen des SNe~Ia avec le redshift établie
dans~\cite{rigault2020}, et pour cela s'est attachée à étudier le paramètre
d'étirement des SNe~Ia qui en est une propriété intrinsèque. La première partie
de ces trois ans de recherche a porté sur la confection d'un échantillon de
données permettant cette analyse, c'est-à-dire qui soit exempt de biais de
sélection et représente tous les étirements qu'il soit possible d'observer.

Cet échantillon a permis d'établir deux modèles de distribution d'étirement
différents selon l'âge des SNe~Ia amenant à une dérive de l'étirement moyen des
SNe~Ia \textit{via} l'évolution des fractions respectives entre vieilles et
jeunes SNe~Ia. Cette approche a ensuite été testée avec d'autres modèles,
dérivants ou non, et nous ont amenæ à montrer que tout modèle non-dérivant était
automatiquement exclu à 5$\sigma$ comme étant de bonnes représentations des
données par rapport à notre modèle. Ce travail a été accepté pour publication le
21 février 2021 dans le journal \textit{Astronomy and Astrophysics}
\citep{nicolas2021}.

La seconde partie de notre thèse a été la prise en main des logiciels \snana\
afin d'intégrer ce modèle dans ce \textit{pipeline} d'analyse cosmologique, de
la confection des corrélations entre SN et galaxie au calcul des modules de
distance corrigés permettant la détermination de paramètres cosmologiques. Ce
travail a permis de mettre en lumière l'existence d'un biais aux alentours de
4\% sur la mesure de $w$ si l'âge d'une SN est en effet le paramètre à l'origine
des variabilités intrinsèques des SNe~Ia. Ce travail constitue le cœur d'un
second article en rédaction.

Ainsi, au travers de ces résultats, nous suggérons aux différentes analyses
cosmologiques se basant sur des simulations et des distributions sous-jacentes
d'étirement d'utiliser le modèle évoluant avec le redshift du
Chapitre~\ref{ch:stretch}, et mettons en garde la communauté quant à
l'utilisation de traceurs moins performant que le LsSFR.

\bibliographystyle{../main/aa_url}
\shorthandoff{:}
\bibliography{../chapters/99_references}

\end{document}
