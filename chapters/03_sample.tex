\documentclass[../main/main.tex]{subfiles}
\begin{document}

\chapter{Cr\'eation d'un \'echantillon complet}\label{ch:sample}

\epigraph{Citation}{Autaire}

Afin d'améliorer l'état de l'art dans le domaine de la cosmologie
observationnelle à l'aide de SNe~Ia et en vue de tester l'évolution de leurs
propriétés avec le reshift, la première étape de ce projet a été de choisir
l'échantillon d'étude.

%\dominitoc
%\tableofcontents
\minitoc
\newpage

\section{Notion de complétude}\label{sec:compl}

Cette thèse repose sur l'étude statistique des propriétés des SNe~Ia, et repose
donc en premier lieu sur l'échantillon de données sur lesquelles développer
notre raisonnement. Pour qu'il soit intéressant il doit être suffisamment grand,
mais également représentatif de la population des SNe~Ia, c'est-à-dire au plus
proche d'être un tirage aléatoire de tout ce qu'il peut exister comme SNe~Ia
dans la nature. On parle alors d'échantillon complet. Ce concept est donc
largement dépendant de la manière dont les données sont relevées.

\subsection{Stratégies d'observations}\label{ssec:startobs}

Les supernovae sont des phénomènes transitoires, c'est-à-dire des objets dont le
flux lumineux varie dans le temps, mais qui sont également brefs et rares~:
elles durent typiquement quelques semaines et surviennent environ une fois par
siècle et par galaxie. Leur observation requiert donc des stratégies
particulières. Pour déterminer leurs courbes de lumière (\ref{ssec:lc}) il est
nécessaire d'avoir un champ de mesure suffisamment profond pour ne pas se
contenter que de leur luminosité au maximum. Différentes approches peuvent
entrer en jeu~: les recherches ciblées et les recherches non ciblées.

\paragraph*{Les recherches ciblées} consistent à se focaliser sur des amas de
galaxies connus en vue d'augmenter la probabilité d'observer des supernovae~; il
paraît en effet évident que plus la concentration en étoiles est forte, plus on
s'attend à avoir une haute probabilité que certaines d'entre elles entament leur
fin de vie et leur explosion en supernova. Cependant, une telle pratique
implique une sélection des environnements des SNe et donc un biais sur la nature
des données recueillies~; dans le cas des amas de galaxies, l'environnement
favorisé sera celui contenant des progéniteurs vieux, dans des galaxies massives
avec peu de formation stellaire. Afin d'étudier la potentielle évolution de la
population des SNe, il faut réduire au maximum ces biais et favoriser la réculte
d'un échantillon représentatif de toute la zoologie des SNe~Ia.

\paragraph*{Les recherches non-ciblées} utilisent de grands champ de caméra pour
sonder de larges portions du ciel. Originellement
\citep[SCP,][]{perlmutter1999}, leur procédé était d'effectuer une
détection photométrique avant d'opérer une identification spectroscopique,
confirmant leur caractère de SN~Ia ou non, pour finalement décider de programmer
ou non un suivi photométrique permettant l'établissement de leur courbe de
lumière. Une telle pratique limite les biais mais donne des courbes de lumières
pauvres en points de mesure avant le maximum de luminosité, impactant
l'ajustement des courbes. Ces méthodes ont évolué pour devenir des recherches
\textit{glissantes}~\citep{astier2006}. Elles consistent à balayer régulièrement
le ciel en observant un même champ dans un même filtre de manière répétée tous
les quelques jours, afin d'à la fois détecter et extraire les courbes de
lumières des SNe~Ia, même si leur identification est effectuée après leur
maximum de luminosité.

\subsection{Biais de Malmquist et solution}\label{ssec:malm}

De tels sondages ne sont cependant pas exempts d'effets de sélection. En effet,
même une recherche glissante s'effectue avec un appareil de mesure ayant une
capacité limitée à détecter une source lumineuse~: les objets de magnitude
apparente plus élevée (luminosité plus faible) que ce seuil de détection ne
seront pas inclus. Or, comme chaque astre voit sa luminosité décroître avec le
carré de la distance qui le sépare de l'observation (\ref{ssec:dl}), cette
limite implique que les astres de magnitude absolue plus élevée seront relevés à
de plus grandes distances que les autres, laissant croire qu'à partir d'une
certaine distance les objets sont intrinsèquement plus lumineux.

Dans le cadre des SNe~Ia dont on suppose la magnitude absolue similaire, on
pourrait en première approche négliger cet effet. Cependant, comme exposé en
Section~\ref{ssec:corr}, il a été déterminé que la magnitude absolue des
supernovae de type Ia est corrélée avec leur \textit{stretch} et leur couleur de
telle sorte que les plus faibles soient celles de petit stretch et de couleur
rouge. Ainsi, proche du seuil de détection, les SNe~Ia ne sont pas sélectionnées
de manière homogène, et l'échantillon recueilli sera une sous-population
laissant penser qu'avec la distance, les SNe~Ia ont en moyenne un plus haut
stretch et sont de couleur bleue. De tels sondages sont dits à magnitude
limitée.

Le cadre de notre étude nécessite un échantillon qu'on appelle
«~volume-limité~», pour lequel on suppose que la population résulte bien d'un
tirage aléatoire de ce qui existe dans la nature. Les sondages modernes reposant
sur des recherches glissantes, il nous a donc fallu les réduire pour les
utiliser.

\section{Présentation des sondages}\label{sec:surveys}
On présente dans cette section les différents sondages utilisés dans notre
étude, principalement non-ciblés.

\subsection{The Nearby Supernova factory}\label{ssec:snf}
\subsubsection{Introduction}\label{sssec:snfintro}

La collaboration \textit{The Nearby Supernova factory}
\citep[SNfactory,][]{aldering2002} est créée peu de temps après la découverte de
l'expansion accélérée de l'Univers \citep{riess1998, perlmutter1999} avec pour
but un suivi spectro-photométrique d'une précision d'environ 1\% de SNe~Ia
proches. L'objectif est de peupler la partie basse du diagramme de Hubble ($0.03
< z < 0.08$), qui ne contenait alors qu'une vingtaine de SNe~Ia
\citep{hamuy1996}, permettant une meilleure détermination de la constante de
Hubble $H_0{}^2$. La faible distance du sondage permet d'éviter d'appliquer des
corrections photométriques dues au décalage vers le rouge (corrections $K$, voir
Section \textbf{?}). La mission tente également d'étudier précisément les
propriétés des SNe~Ia grâce au traceur LsSFR (Section \textbf{?}) afin de mieux
comprendre leur diversité, mettre en évidence différentes populations de
supernovae et améliorer leur standardisation grâce à une meilleure compréhension
de leurs variabilités et ainsi réduire les erreurs systématiques dans les
mesures de paramètres cosmologiques.

\subsubsection{Détection des supernovae}\label{sssec:snfdetec}

Le programme a été sujet à plusieurs évolutions au cours de son fonctionnement,
notamment pour la découverte de nouveaux candidats. Ce sont d'autres télescopes
qui alertent la communauté. En premier lieu, jusqu'à fin 2008, le télescope de
\SI{1.2}{m} du mont Palomar en Californie \citep{rabinowitz2003} scannait
\SI{500}{deg^2} du ciel chaque soir avec la caméra QUEST de 112 capteurs CCD. À
partir de 2010, les canditats de SNe proviennent d'une coopération avec
\textit{Palomar Transient Factory} \citep[PTF,][]{law2009} et de données
publiques. La caméra QUEST fut ensuite déplacée à La Silla au Chili
\citep[LSQ,][]{hadjiyska2012} pour reprendre, mi-2012, l'activité de recherche
de SNe pour SNfactory. Les candidats potentiels sont à chaque fois programmés
pour observation spectroscopique afin de les identifier en tant que SNe~Ia et
décider de leur suivi selon des critères de qualité (nombre de points de mesure,
proche et avant du maximum, non-contamination pas la luminosité de la Lune
notamment).

\subsubsection{Suivi spectro-photométrique}\label{sssec:snfspectro}

Le typage spectroscopique, quand il n'a pas déjà été réalisé par d'autres
collaboration ayant donné l'alerte, est assuré par le \textit{SuperNovae
Integral Field Spectrograph} \citep[SNIFS,][]{lantz2004} du télescope de
l'Université d'Hawaii de \SI{2.2}{m} au sommet du Mauna Kea, mis en service en
2004. Il s'avère plus efficace qu'un typage photométrique qui nécessite
plusieurs observations dans différents filtres de couleur, bien que ces
dernières soient plus simple à mettre en place.

Ce spectrographe dit «~à champ intégral~» récolte des «~cubes~», des données en
3 dimensions, deux spatiales représentant un point dans le ciel plus une
dimension de longueur d'onde~; chaque point de ce relevé se nomme
\textit{spaxel}, pour «~spatial picture element~», et ensemble forment une
grille de 15 $\times$ 15 pour un champ de vue total de \ang{;;6.4} $\times$
\ang{;;6.4} dans deux longueurs d'ondes~: une voie bleue ($B$) de 3200 à
\SI{5200}{\angstrom} et une voie rouge ($R$) de 5100 à \SI{10000}{\angstrom}.

En plus de cette voie, SNIFS possède une voie photométrique utilisant 5 filtres
$ugriz$ pour suivre l'absorption atmosphérique, et une voie de guidage avec un
filtre $V$ pour aider le télescope à la focalisation. Le champ de ces caméras
est de \ang{;4.5;} $\times$ \ang{;9;}.

\subsubsection{Taux de formation stellaire spécifique
spectroscopique}\label{sssec:snflssfr}

La spécificité de SNIFS est de permettre des mesures spectroscopiques de
l'environnement immédiat des SNe, développée dans~\cite{rigault2013} et résumée
dans~\cite{rigault2020}. Ce procédé commence par modéliser le spectre du ciel
que l'on soustrait aux cubes avant d'extraire le spectre de l'environnement dans
un rayon de \SI{1}{kpc} projeté autour de la position des SNe~Ia. Ces données
permettent de détecter l'émission de raies H$\a$, l'un des indicateurs
traditionnellement les plus utilisés pour mesurer le taux de formation stellaire
(\textit{stellar formation rate}, SFR~; cf.~\cite{kennicutt1998}).

\NN{Description principe}

\subsubsection{Description des données conservées}\label{sssec:snfdata}

De 2004 à 2013, SNfactory a classifé 1364 objets dont plus de 1000 supernovae,
observé 645 SNe~Ia au moins une fois et en a suivi plus de 271 SNe~Ia, avec au
moins 5 points de mesure \citep{copin2013}.

Sur celles-ci, 198 ont des mesures satisfaisant les contraintes nécessaires à
l'établissement de leur courbe de lumière et sont associées à une galaxie hôte
permettant de déterminer leur redshift~; pour efficacement déterminer les
propriétés locales de leur environnement, seules les SNe entre $z = 0.02$ et $z
= 0.08$ sont conservées, amenant l'échantillon à 160 objets.

Les données pour lesquelles les images des galaxies hôtes dans les bandes
photométriques $g$ et $i$ sont contaminées par la luminosité des SNe sont
également rejetées~; ces bandes s'avèrent en effet nécessaires à la
détermination de la masse stellaire de la galaxie. Cette coupe réduit
l'échantillon à 147 objets.

Les SNe considérées comme trop «~anormales~» sont exclues car supposées non
représentatives de la population générale que l'on souhaite étudier. Elles sont
au nombre de 6.

Finalement, parmi ces 141, ne sont conservées que celles provenant directement
des collaborations internes \NN{ou uniquement celles SNIFSées ?}, c'est-à-dire
celles de SNf, PTF et LSQ. L'échantillon final est alors de 114 données.  
L'ensemble de ces critères de sélection est résumé Table~\ref{tab:snfcuts}.

Grâce au suivi spectroscopique de tous les candidats à $r \lesssim
\SI{19.5}{mag}$ et ces limitations en redshift, ces données sont considérées
comme étant limitées en volume, c'est-à-dire un tirage aléatoire des populations
sous-jacentes de SNe~Ia.

\begin{table}[]
    \centering
    \caption{Critères de sélection des SNe~Ia suivies par SNfactory.}
    \label{tab:snfcuts}
    \begin{tabular}{lc}
        \toprule
        Critères de sélection           & Nb de SNe~Ia \\
        \midrule
        Suivies                         & 271 \\
        Courbe de lumière + hôte        & 198 \\
        $0.02 < z < 0.08$               & 160 \\
        Hôte $g$ et $i$ non contaminées & 147 \\
        SNe~Ia «~normales~»             & 141 \\
        SNf LSQ ou PTF seulement        & 114 \\
        \bottomrule
    \end{tabular}
\end{table}

\subsection{Sloan Digital Sky Survey}\label{ssec:sdss}
\subsubsection{Introduction}\label{sssec:sdssintro}

Le sondage \textit{Sloan Digital Sky Survey} \citep[SDSS,][?]{sako2008} est un
programme d'observation astronomique majeur qui a débuté en 2000 et est encore
actif aujourd'hui. Le programme se divise en cinq phases d'observation de
différents objets astrophysiques, de simples étoiles aux grandes structures de
l'Univers. La partie supernova du sondage se limite à la seconde phase entre
2005 et 2008, qui sera la seule que nous détaillerons ici. L'intervalle de
redshift sondés est entre $0.05 \lesssim z \lesssim 0.4$, partie encore peu
sondée à l'époque par les autres relevés.

\subsubsection{Détection des supernovae}\label{sssec:sdssdetec}

La stratégie d'observation de SDSS se concentre sur \SI{300}{deg^2} du ciel en y
répétant l'acquisition. Elle est réalisé grâce au télescope optique dédié de
\SI{2.5}{m} \citep{gunn2006} à Apache Point au Nouveau Mexique, couplé à une
caméra CCD \citep{gunn1998} à 5 filtres optiques
\citep[$ugriz$,][]{fukugita1996} qui tournent avec une cadence relativement
haute, entre 4 et 5 acquisition par nuit. Le procédé est similaire à celui de
SNf, étant tous les deux des sondages à recherche glissante~: différentes images
du ciel sont comparées pour détecter les phénomènes transitoires et créer des
débuts de courbes de lumière.

Pour discriminer entre bruit de fond et réelle variation astronomique, une
inspection visuelle par un humain était systématiquement nécessaire jusqu'en
2006, \textit{via} une interface web comportant les images dans les filtres
$gri$ et d'autres informations pertinentes. Après cette date, la détection est
en partie laissée au logiciel \textit{autoscanner}. Sur les trois saisons
d'observation, ce sont 10258 nouveaux objets transitoires qui ont été
découverts.

\subsubsection{Suivi spectro-photométrique}\label{sssec:sdssspectro}

Le typage de SDSS utilise de nombreux différents télescopes~: le HET de
\SI{9.2}{m}, le NTT de \SI{3.6}{m}, le ARC de \SI{3.5}{m}, Subaru de
\SI{8.2}{m}, le MDM Hiltner de \SI{2.4}{m}, le WHT de \SI{4.2}{m}, le KNPO
Mayall de \SI{4}{m}, le Keck de \SI{10}{m}, le NOT de \SI{2.5}{m} et le SALT de
\SI{11}{m}~; plusieurs de ces télescopes pouvaient être prévus pour observation
la même nuit, rendant au total le temps alloué à la spectroscopie supérieur au
temps alloué à l'acquisition optique.

Cependant, le nombre de candidat par nuit excède largement les capacités de
suivi spectroscopique, obligeant les opérateurs à faire une sélection des cibles
à analyser. Ainsi, un typage photométrique a également été réalisé en comparant
les courbes de lumière dans les bandes $gri$ avec des librairies de différents
\NN{template} pour en estimer les paramètres (redshift, date du maximum de flux,
magnitude apparente, contamination galactique...) et permettre de prioriser les
cibles à suivre spectroscopiquement. L'algorithme choisi par SDSS se rapproche
fortement de celui du sondage SNLS, cf Section~\ref{ssec:snls}.

\subsubsection{Données conservées}\label{sssec:sdssdata}

Le sondage requiert généralement deux détections avant le suivi spectroscopique,
mais par manque de ressources ce critère a pu être réduit. 

Data: Sako 2014 ; télescope Gunn 2006 ; camera Gunn 1998 ; filtres $ugriz$
Fukugita 1996.

\subsection{Panoramic Survey Telescope and Rapid Response
System}\label{ssec:ps1}

\subsection{SuperNovae Legacy Survey}\label{ssec:snls}



\subsection{Hubble Space Telescope}\label{ssec:hst}

Ces données à haut redshift, inclues dans Pantheon, se révèlent d'une grande
importance par leur poids dans le diagramme de Hubble pour tester l'évolution
des propriétés des SNe. À ces distances, le typage peut s'avérer difficile mais
la classification était suffisamment robuste pour les inclure dans l'analyse
cosmologique \citep{scolnic2018}~; ces données ne sont donc pas sujettes aux
coupes en redshifts de la Section~\ref{ssec:cuts}.

\subsection{Autres sondages : CfA1-4 et CSP}\label{ssec:lowz}


\section{Échantillon d'étude}\label{sec:sample}

Nous détaillons dans cette Section la procédure de construction de notre
échantillon volume-limité comme expliqué partie~\ref{ssec:malm}. Notre étude se
base sur les données de la combinaison de sondage Pantheon \citep{scolnic2018},
en remplaçant la combinaison ciblée LOWZ par les données SNfactory dont la
sélection est maîtrisée et permettant une étude de sous-population grâce au
LsSFR. Les données de HST étant complètes, la confection de notre échantillon se
concentre sur les sondages SDSS, PS1 et SNLS~; leur nature non-ciblée et limitée
en magnitude permet d'en construire une portion limitée en volume.

\subsection{Confection}\label{ssec:cuts}

Nous détaillons ici deux des approches mises en place visant à déterminer la
portion des sondages que l'on peut considérer limitées en volume.

\subsubsection{Approche statistique}\label{sssec:baserate}

À partir des données publiées dans \cite{scolnic2018}\footnote{\href{
https://archive.stsci.edu/hlsps/ps1cosmo/scolnic/data_fitres/}
{https://archive.stsci.edu/hlsps/ps1cosmo/scolnic/data\_fitres}}, il est
possible de tracer l'histogramme des SNe~Ia en fonction du redshift~; un exemple
est montré Figure~\ref{fig:zmax_method} pour les données de SNLS. En supposant
une densité volumique de supernovae uniforme, chaque intervalle de redshift
comprend un volume de plus en plus grand et on s'attend donc à observer toujours
plus de SNe~Ia avec la distance.

\begin{figure}[]
    \centering
    \begin{subfigure}[]{.49\linewidth}
        \centering
        \includegraphics[width=\linewidth]{zmax_method_snls-01.pdf}
        \captionsetup{justification=centering}
        \caption{11 intervalles, ajustement sur 6}
        \label{fig:zmax_method1}
    \end{subfigure}
    \begin{subfigure}[]{.49\linewidth}
        \centering
        \includegraphics[width=\linewidth]{zmax_method_snls-02.pdf}
        \captionsetup{justification=centering}
        \caption{7 intervalles, ajustement sur 6}
        \label{fig:zmax_method2}
    \end{subfigure}
    \captionsetup{justification=centering}
    \caption{Exemple d'ajustement statistique pour deux tirages aléatoires
    d'histogrammes de SNLS}
    \label{fig:zmax_method}
\end{figure}

La chute de nombre de SNe~Ia provient de cette limitation du sondage à mesurer
la luminosité. Notre première approche a été de se baser sur une étude
statistique pour essayer de récupérer la valeur estimée à partir de laquelle
chaque sondage s'écarte d'un modèle volumétrique. Le protocole est le suivant~:
\begin{itemize}
    \item Les bornes minimales et maximales des données sont augmentées d'une
        faible valeur aléatoire (entre 0.06 et 0.12 à gauche et entre 1.10 et
        1.15 à droite, pour SNLS) afin d'assurer une variation du centre des
        intervalles~;
    \item On choisit aléatoirement entre 5 et 20 intervalles pour tracer
        l'histogramme~;
    \item On initialise un modèle volumétrique $a\times
        \left(V(z_2)-V(z_1)\right)$ avec $a$ la densité volumique de SNe~Ia,
        paramètre libre du modèle, auquel on passe comme donnée les bords des
        intervalles.
    \item Les valeurs du modèle sont comparées aux hauteurs des
        intervalles de l'histogramme, permettant l'ajustement du modèle par une
        loi de Poisson cumulée~:
        \begin{equation}\label{eq:poisson}
            P(x<X) = \lambda e^(x/b)
        \end{equation}
    \item On choisit aléatoirement un intervalle maximal après lequel
        l'ajustement s'arrête, avec un minimum de 3 intervalles (cas de la
        Figure~\ref{fig:zmax_method1}), 10 fois pour chaque histogramme~;
    \item On sauve les positions et valeurs de probabilité des intervalles
        ajustés et créé une interpolation linéaire des résultats~;
    \item Ces 5 étapes sont répétées 1000 fois et on calcule la médiane et
        l'écart type des 10 000 interpolations calculées.
\end{itemize}

% Deux modèles ont été testés. Le premier est issu de \textbf{perrett2012} donnant
% un taux volumétrique co-mobile~:
% \begin{equation}\label{eq:perrett}
%     R(z) = 1.75\times10^{-5}(1+z)^{2.11}\,{\rm yr^{-1}Mpc^{-3}}
% \end{equation}
% Le second est directement issu du calcul d'un volume co-mobile dérivé des
% équations de la relativité générale~:

Le modèle retenu dans notre analyse est défini par~:
\begin{equation}\label{eq:comobvol}
    V(z) = \frac{4\pi}{3}\times d_C{}^3(z)
\end{equation}
avec $d_C(z)$ la distance comobile
\begin{align}
    d_C(z)                           & =
    \frac{c}{H_0} \int_{0}^{z} \frac{\d z'}{E(z')}
    \quad\text{avec}\quad\\
    E(z) \triangleq \frac{H(z)}{H_0} & =
    \left[\Omega_R(1+z)^4 + \Omega_M(1+z)^3 +
        \Omega_k(1+z)^2 + \Omega_\Lambda
    \right]^{1/2}
\end{align}
On a choisi la cosmologie issue de la collaboration Planck~\citep{planck2018}~:
\begin{itemize}
    \item $H_0 = \SI{67.74}{km.Mpc^{-1}.s^{-1}}$~;
    \item $\Omega_R = 5.389\times10^{-5}$~;
    \item $\Omega_M = 0.3075$~;
    \item $\Omega_k = 0$~;
    \item $\Omega_\Lambda = 0.6910$
\end{itemize}

Le résultat de ces calculs donne une estimation du redshift à partir duquel
chaque sondage n'a plus la capacité à recueillir toutes les SNe~Ia, représentée
Figure~\ref{fig:zmax_method_results}. En estimant $z_{\lim}$ comme étant la
valeur à laquelle la médiane des distributions cumulées chute à 0.5 et les
erreurs basse et haute à 0.525 et 0.475 respectivement, on obtient les valeurs
de la Table~\ref{tab:zlimsample}.

\begin{SCfigure}[0.5]
    \centering
    \includegraphics[width=.7\linewidth]{zmax_method_results.pdf}
    \captionsetup{justification=centering}
    \caption{Résultat graphique de l'évolution médiane de l'étude statistique du
    redshift limite pour les sondages SDSS, PS1, et SNLS}
    \label{fig:zmax_method_results}
\end{SCfigure}

Cette première approche présente une robustesse certaine dans l'établissement
des évolutions statistiques en répétant le processus précédent. Cependant, le
sens de variation non constant du résultat de SNLS et de PS1 ne permettent pas
de forte confiance dans la correspondance de ce protocole au but de cette
étude~; de plus, le choix de la valeur de la fonction de répartition à laquelle
on peut considérer le sondage complet n'est pas motivée mathématiquement ou
physiquement de manière systématique. Cette conclusion nous a amenæ à une
approche combinant à la fois la réalité de la sélection astrophysique
instrumentale et les équations de distribution de luminosité de SN~Ia avec leurs
paramètres $x_1$, $c$ et $z$.

\subsubsection{Approche analytique}\label{sssec:maglim}

En supposant que ces sondages ont un typage spectroscopique et suivi
photométrique suffisant, ils devraient avoir des effets de sélection de
sous-population de SNe~Ia négligeables en deçà d'un certain redshift permettant
l'acquisition de toute la zoologie de stretch et couleur. Les données de SNe~Ia
issues de l'ajustement par \texttt{SALT2.4} ne contiennent que des données avec
un maximum de $x_1 = \pm 3$ et de $c = \pm 0.3$ \citep[][cf
Section~\ref{ssec:salt}]{guy2007, betoule2014}.

La magnitude absolue d'une supernova à son maximum de luminosité est, d'après
l'équation~\ref{eq:mxc}~:

\begin{equation*}
    M = M_0 -\alpha x_1 + \beta c
\end{equation*}
avec $M_0 = \SI{-19.36}{mag}$ dans le filtre photométrique $B$ de Bessell
\citep{kessler2009a, scolnic2014}, $\alpha=0.158$ et $\beta=3.14$
\citep[Table 7,][]{scolnic2018}. On détermine cette quantité sur l'ellipse limite
des paramètres grâce au paquet \texttt{sncosmo}
\footnote{\href{https://sncosmo.readthedocs.io/en/stable/}
{https://sncosmo.readthedocs.io/en/stable/}}, représentée par un gradient de
couleur Figure~\ref{fig:maglim}. On trouve alors que la supernova la moins
lumineuse est celle de paramètres $x_1 = -1.65$ et $c = 0.25$ dont le maximum de
magnitude absolue standardisée est $M_{\min}^{t_0}=\SI{-18.31}{mag}$.

\begin{figure}
    \centering
    \includegraphics[width=0.95\linewidth]{zmax_maglim_snls.pdf}
    \caption{Distribution des paramètres de courbe de lumière de stretch ($x_1$)
        et de couleur ($c$) issus d'un ajustement par \texttt{SALT2.4} pour les
        données de SNe~Ia des sondages SDSS, PS1 et SNLS combinés du catalogue
        Pantheon. Chaque supernova est représentée par un point bleu. L'ellipse
        limite des paramètres $(x_1=\pm3, c=\pm0.3)$ est représentée avec un
        gradient de couleur correspondant à la magnitude absolue standardisée en
        utilisant les valeurs de \cite{scolnic2018} pour les coefficients
        $\alpha$ et $\beta$. Les lignes diagonales grises représentent
        l'évolution de $m = m_{\lim}$ en fonction de $z$ dans le plan $(x_1,c)$
        entre $z=0.50$ et $z=1.70$ pour la magnitude limite
    $m_{\lim}=\SI{24.8}{mag}$ du sondage SNLS.}
    \label{fig:maglim}
\end{figure}

Cependant, pour établir une courbe de lumière, une supernova doit être observée
typiquement au moins 5 jours avant et une semaine après son pic de luminosité,
donnant une magnitude absolue limite effective d'approximativement $M_{\lim}
= \SI{-18.00}{mag}$. En connaissant les magnitudes limites de chaque sondage et
avec l'équation reliant le module de distance aux magnitudes observée et
absolue
\begin{equation}\label{eq:distmod}
    \mu(z) = m - M
\end{equation}
on peut déterminer le redshift limite $z_{\lim}$ en-delà duquel la SN~Ia la
moins lumineuse ne sera pas observée. On a ainsi défini un ensemble de redshifts
limites définissant un échantillon fiduciel en prenant la limite suggérée par
cette analyse.

Cependant, cette solution pourrait ne pas être optimale étant donné qu'elle
ignore les efficacités de suivi spectroscopiques pour les redshifts en-dessous
de $z_{\lim}$~; c'est pourquoi nous avons également déterminé un autre ensemble
de coupes définissant un échantillon «~conservatif~». Cet échantillon est plus
petit et donc sera statistiquement moins pertinent, mais également moins sujet
aux effets de sélection. Ainsi si l'évolution des propriétés des SNe~Ia avec le
redshift est encore sondable dans l'échantillon conservatif, il serait encore
plus présent dans un échantillon dont l'absence d'effets de sélection est
effectuée avec plus de précision que nos coupes en redshift.

\paragraph*{Pour SNLS} dont les supernovae sont typiquement entre $0.4 < z <
0.8$, la bande $B$ de Bessell dans un référentiel au repos correspond
approximativement à son filtre $i$, de magnitude limite à 5$\sigma$ de
\SI{24.8}{mag}\footnote{\href{
    https://www.cfht.hawaii.edu/Science/CFHTLS/cfhtlsfinalreleaseexecsummary.html}
{https://www.cfht.hawaii.edu/Science/CFHTLS/cfhtlsfinalreleaseexecsummary.html}}.
Ceci implique $z_{\lim} = 0.60$, en accord avec \cite{neill2006, perrett2010},
et \cite{bazin2011}. D'autre part, la Figure~14 de \cite{perrett2010}, suggère
une plus basse limite à $z_{\lim} = 0.55$. Nous avons donc choisi $z=0.60$ et
$z=0.55$ comme redshifts limites de SNLS pour les échantillons fiduciel et
conservatif respectivement.

\paragraph*{De la même manière pour PS1} leurs SNe~Ia sont entre $0.2 < z <
0.4$~; la profondeur à 5$\sigma$ dans la bande $g$ est de \SI{23.1}{mag} d'après
\cite{rest2014} et mène à $z_{\lim}=0.31$, en correspondance avec la Figure~6
de \cite{scolnic2018} par exemple. De manière conservative, cette figure
suggère une limite plus prononcée à $z_{\lim}=0.27$~; ces deux valeurs
constituent donc les redshifts limites de PS1 pour la partie fiducielle et
conservative, respectivement, de notre échantillon.

\paragraph*{Dans le même intervalle pour SDSS} la magnitude limite est de
\SI{22.5}{mag} d'après \cite{dilday2008} et \cite{sako2008}~; cette valeur 
impliquerait $z_{\lim}=0.24$, mais les sondages SDSS se sont confrontés à une
limitation dans leurs capacités spectroscopiques. Comme indiqué dans
\cite{kessler2009a} Section~2, les données de la première année de SDSS ont
favorisé les SNe~Ia de magnitude $r < \SI{20.5}{mag}$ pour identification
spectroscopique, ce qui correspondrait à une coupe de redshift à 0.15. Le reste
du programme a bénéficié de meilleures ressources spectroscopiques et
\cite{kessler2009a} et \cite{dilday2008} font preuve d'une complétude
raisonnable jusqu'à $z=0.2$. En nous basant sur ces faits, on a choisi
$z_{\lim}=0.20$ et $z_{\lim}=0.15$ pour nos échantillons fiduciel et
conservatif respectivement.

Cette approche est totalement systématique et reproductible, \textbf{insert
github pages notebook?} et donne des $z_{\lim}$ similaires à l'approche
statistique~; cette observation conforte donc les résultats et choix de
magnitude limite, et ce sont ces résultats analytiques que l'on a conservés dans
notre étude.

\subsection{Présentation}\label{ssec:dataset}

\begin{table}[]
    \centering
    \begin{threeparttable}
        \caption{Composition en SNe~Ia de notre échantillon.}
        \label{tab:zlimsample}
        \begin{tabular}{lccc}
            \toprule
            \multirow{2}[3]{*}{Sondage} &
            \multicolumn{2}{c}{$z_{\lim}$} &
            \multirow{2}[3]{*}{$N_{\rm SN}$}\\
            \cmidrule(lr){2-3}
            & Statistique & Analytique & \\
            \midrule
            SNf &
            \multicolumn{2}{c}{0.08} &
            114 \\
            SDSS & 
            0.204$^{+0.001}_{-0.001}$ & 0.20 (0.15) &
            167 (82) \\
            PS1 &
            0.284$^{+0.006}_{-0.008}$ & 0.31 (0.27) &
            160 (122) \\
            SNLS &
            0.615$^{+0.003}_{-0.003}$ & 0.60 (0.55) &
            102 (78) \\
            HST &
            \multicolumn{2}{c}{--} &
            26 \\
            \midrule
            Total & \multicolumn{2}{c}{--} &
            569 (422)\\
            \bottomrule
        \end{tabular}
        \begin{tablenotes}[flushleft]
        \item \textbf{notes.} L'échantillon et notamment le nombre de SNe
            utilisées suivent les limites analytiques. Les nombres entre
            parenthèse correspondent aux limites conservatives.
        \end{tablenotes}
    \end{threeparttable}
\end{table}

La composition finale de notre échantillon est synthétisée
Table~\ref{tab:zlimsample}. Les distributions en redshift des 3 sondages coupés
sont présentées Figure~\ref{fig:cuts}. On y observe que les limites sont
globalement situées avant le pic de ces histogrammes, suivant la logique guidant
cette chute (cf. Section~\ref{sec:compl}).

\begin{figure}
    \centering
    \includegraphics[width=0.80\linewidth]{hist_surveys_cuts}
    \caption{\textit{De haut en bas}: Histogrammes en redshift des SNe~Ia des
        sondages SDSS, PS1 et SNLS \citep[données de Pantheon,][]{scolnic2018}.
        Les parties colorées représentent les distribution de SNe~Ia conservées
        dans notre analyse, considérées exemptes d'effets de sélection
        observationnels (cf. Section~\ref{sssec:maglim}). Les couleurs foncées
        (claires) représentent les limites conservatives (fiducielles) nos
    coupes de sélection indiquées dans la Table~\ref{tab:zlimsample}.}
    \label{fig:cuts}
\end{figure}

En combinant les 5 sondages de notre analyse, on peut tracer leur distribution
de stretch en fonction du redshift. On en présente un graphique
Figure~\ref{fig:sample}. Supposant l'échantillon affranchi d'effets de
sélection, on peut lire sur ce graphique une première idée de l'évolution en
redshift que l'on suppose issue du changement des propriétés moyennes des SNe~Ia
avec l'âge de leur environnement. En effet, on observe que la fraction de SNe~Ia
présentant un faible stretch, typiquement $x_1 < -1$, semble décroître avec le
redshift alors que la population de stretch $> 1$ semble toujours peuplée~; à
noter qu'ici le redshift est en échelle logarithmique, expliquant le tassement
horizontal. Cette idée est confirmée dans le chapitre suivant,
Section~\ref{sec:xres}.

\begin{figure}
    \centering
    \includegraphics[width=0.95\linewidth]{stretchs-cut_btw_hist_stac.pdf}
    \caption{\textit{En bas}~: stretch des courbes de lumière ajustées avec
        \textsc{\texttt{SALT2.4}} en fonction du redshift en échelle
        logarithmique pour chaque sondage de cette analyse (cf.~légende). Les
        points pleins (creux) correspondent aux limites conservatives
        (fiducielles). \textit{En haut}~: histogrammes en redshift superposés,
        en sombre (clair) pour les limites conservatives
    (fiducielles).}\label{fig:sample}
\end{figure}


\subsection{Confirmation d'hypothèse}\label{ssec:testvl}

\bibliographystyle{../main/aa_url}
\shorthandoff{:}
\bibliography{../chapters/99_references}

\end{document}
