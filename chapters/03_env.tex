\documentclass[../main/main.tex]{subfiles}
\begin{document}

% \dominitoc
% \faketableofcontents
% \dominilof
% \fakelistoffigures
% \dominilot
% \fakelistoftables

\chapter{Variabilit\'es environnementales des SNe~Ia}\label{ch:env}
\epigraph{\openquote Citation\closequote}{\textsc{Autaire}, \textit{Livre}}

Standardisation pas suffisante. Mesure de $H_0$ contestée, calibrations
différentes, réduction systématique par environnement.

Définition galaxie,

Morphologie, masse, SFR

\vfill
\minitoc
\vfill
\newpage

\section{Présentation d'environnements galactiques}\label{sec:envpres}
\subsection{Morphologie}\label{ssec:morphost}
\subsection{Couleur}\label{ssec:chost}
\subsection{Masse}\label{ssec:mhost}
\subsection{Taux de formation stellaire}\label{ssc:sfrhost}

\section{Corrélation avec la masse}\label{ssec:mcorr}
\subsection{Corrélation en étirement}\label{sssec:mcorrx1}
\subsection{Marche de magnitude basée sur la masse}\label{sssec:mstep}
\subsection{Implication en cosmologie moderne}\label{sssec:mcosmo}

\section{Notion d'âge}\label{ssec:spectro}
\subsection{Distribution du temps de retard}\label{sssec:dtd}
\subsection{Taux de formation stellaire spécifique
spectroscopique et âge}\label{ssec:lssfr}

La spécificité de SNIFS est de permettre des mesures spectroscopiques de
l'environnement immédiat des SNe, développée dans~\cite{rigault2013} et résumée
dans~\cite{rigault2020}. Ce procédé commence par modéliser le spectre du ciel
qui est soustrait aux cubes avant d'extraire le spectre de l'environnement dans
un rayon de \SI{1}{kpc} projeté autour de la position des SNe~Ia. Ces données
permettent de détecter l'émission de raies H$\a$, l'un des indicateurs
traditionnellement les plus utilisés pour mesurer le taux de formation stellaire
(\textit{stellar formation rate}, SFR~; cf.~\cite{kennicutt1998}), en
l'occurrence dans l'environnement local (à moins de \SI{1}{kpc}). Il repose sur
le fait que les étoiles massives ($\gtrsim \SI{20}{\Msun}$) génèrent des photons
ultra-violets (donc à haute énergie) capables d'ioniser les gaz d'hydrogène de
leur environnement \citep{calzetti2013} en grande quantité. Ces atomes excités
vont ensuite se recombiner, produisant diverses raies d'émission dont certaines
dans la série de \textsc{Balmer}. C'est ce principe qui fournit les raies H$\a$,
dont la longueur d'onde dans le vide est $\lambda_{\mathrm{H}\a} =
\SI{656.5}{nm}$. L'étude de cette raie permet l'estimation de la formation
stellaire du fait que les étoiles massives ont une courte durée de vie, à
l'échelle de millions d'années, et que leur capacité à générer de tels photons
décroît très rapidement~: le flux généré décroît de deux ordres de grandeurs
en approximativement \SI{10}{Mans}. La présence d'hydrogène ionisé est donc un
indicateur direct de la présence de «~jeunes~» étoiles, c'est-à-dire de moins de
\SI{100}{Mans}, et du taux de formation stellaire \textit{via} la
correspondance donnée dans~\cite{calzetti2013}:
\begin{equation}\label{eq:sfrha}
    \mathrm{SFR}(\mathrm{H}\alpha)\,[\si{\Msun.an^{-1}}] =
    5,45\times 10^{-42}L(\mathrm{H}\a)\,[\si{erg.s^{-1}}]
\end{equation}
avec $L(\mathrm{H}\a)$ la luminosité des raies d'émission, obtenue par un
ajustement spectral.

En supposant la quantité de vieilles étoiles proportionnelle à la masse
stellaire $M_*$ de la galaxie hôte \citep{mannucci2005, scannapieco2005}, il est
possible de tracer la fraction de jeunes étoiles \textit{via} l'utilisation du
taux de formation stellaire \textit{spécifique}, sSFR, tel que~:
\begin{equation}\label{eq:ssfr}
    \mathrm{sSFR} = \frac{\mathrm{SFR}}{M_*}
\end{equation}
On l'appelle alors «~local~», et on le dénote LsSFR, quand ce ratio calculé dans
un environnement projeté de \SI{1}{kpc} autour de l'astre en question. Cette
approche locale a pour but de déterminer plus précisement l'âge qu'avec des
caractéristiques globales (morphologie, masse stellaire totale…). Une étude
complète de la capacité de ce traceur à déterminer l'âge d'une SNe~Ia a été
effectuée dans~\cite{briday2021, briday2022}.

Il est attendu que la formation stellaire soit plus élevée à haut redshift, au
début de l'histoire de l'Univers, alors que les étoiles vieilles, suivant la
masse de leur galaxie hôte, soit plus élevée à bas redshift. Ainsi, le sSFR est
un ordre de magnitude plus élevé à $z = 1,5$ qu'à $z = 0$ \citep[voir][pour une
étude complète]{madau2015}. En pratique, les mesures de~\cite{tasca2015}
trouvent un dépendance en redshift~:
\begin{equation}\label{eq:zssfr}
    \mathrm{sSFR} \propto (1+z)^{2,8 \pm 0,2}
\end{equation}
Les travaux de~\cite{rigault2020} combinent alors la fraction de jeunes étoiles
($\delta(z)$) et de vieilles étoiles ($\psi(z)$, telle que $\delta(z) + \psi(z)
= 1$) ainsi que l'équation~\ref{eq:zssfr} pour déduire~:
\begin{align}\label{eq:dpz}
    \mathrm{LsSFR}(z) \triangleq \frac{\delta(z)}{\psi(z)} &=
        K\times(1+z)^{\phi} \\\label{eq:deltaz}
    \text{et ainsi}\quad
    \delta(z) & = \left( K^{-1}\times(1+z)^{-\phi} +1 \right)^{-1}
    \\\label{eq:psiz}
    \psi(z)   & = \left( K\times(1+z)^{+\phi} +1 \right)^{-1}
\end{align}
avec $K=0,87$ en fixant $\delta(0,05) = \psi(0,05) = 0,5$ et $\phi = 2,8$.
\subsection{Marche de magnitude basée sur l'âge}\label{sssec:astep}

\newpage

\thispagestyle{plain}
% \vspace*{-3cm}
\vfill
\minilof
\vfill
\minilot
\vfill

% \bibliographystyle{../main/aa_url}
% \shorthandoff{:}
% \bibliography{../chapters/99_references}

\end{document}
